
%-------------------------------------------------------------------------------
%							SECOND SECTION
%-------------------------------------------------------------------------------

\section{\textsc{État de l'art}}


\subsection{Thèse de M. Rabatel}


\begin{frame}{Cinétique du floe}
  \mycols{
    \mycol{50}{
      Les équations de Newton-Euler:
      \begin{align} \label{eq:neweu}
      \begin{dcases} 
        M_i \frac{\diff \dot{\bvec{G}}_i(t)}{\diff t} &= \bvec{F}_i, \\
        \mathcal{I}_i \frac{\diff \dot{\theta}_i(t)}{\diff t} &= \mathfrak{M}_i,
      \end{dcases}        
      \end{align}
      où pour le floe $i$ : 
      \begin{itemize}
        \item $M_i$ : masse du floe;
        \item $\bvec{F}_i$ : somme des forces par unité de volume;
        \item $\mathcal{I}_i$ : le moment d'inertie du floe $i$;
        \item $\mathfrak{M}_i$ : le moment dynamique en $G$.
      \end{itemize}
      Le système (\ref{eq:neweu}) se réécrit sous la forme :
      \begin{align*}    
          \mathcal{M}_i \frac{\diff W_i(t)}{\diff t} = \mathcal{H}_i(t) ,
      \end{align*}
      avec 
      \begin{align*}
      \mathcal{M}_i = 
      \begin{pmatrix}
          M_i & 0 & 0 \\ 0 & M_i & 0 \\ 0 & 0 & \mathcal{I}_i
      \end{pmatrix} , \quad
      W_i(t) = 
      \begin{pmatrix}
          \dot{\bvec{G}}(t) \\ \dot{\theta}_i(t)
      \end{pmatrix} , 
      \text{ et } \quad \mathcal{H}_i(t) = 
      \begin{pmatrix}
          \bvec{F}_i(t) \\ \mathfrak{M}_i(t)
      \end{pmatrix}.
      \end{align*}
    }


    \mycol{50}{
      \myfig{FloeRabatel}{Repères abosolu et local pour un floe}
    }
  }
  
\end{frame}



\begin{frame}{Interaction entre les floes}
  \mycols{

    \mycol{50}{
      \myfig{Collision1.png}{Interaction entre deux floes $\Omega_k$ et $\Omega_l$ au point $P_j$}
    }

    \mycol{50}{Deux conditions à respecter:
    \begin{itemize}
      \item condition unilatérale de Signorini;
      \item loi de friction de Coulomb.
    \end{itemize}}

  }

\end{frame}



\begin{frame}{Discussion sur la thèse}

  \mycols{

    \mycol{55}{
      \begin{itemize}
        \item Les floes sont rigides;
        \item Le modèle ne gère pas la rhéologie de la glace;
        \item Les coefficients de friction et de restitution sont limitants.
      \end{itemize}
    }

    \mycol{45}{
      \myfig{Derive.png}{Dérive dans un canal étroit}
    }

  }

\end{frame}




\subsection{Thèse de D. Balasoiu}


\begin{frame}{Un modèle de fracture variationnel}
  L'énergie totale s'écrit :
  \begin{align*}
  \etot : \,\bigcup_{\sigma \in \Sigma } A_{\sigma} \times \left\{ \sigma \right\} &\rightarrow \mathbb{R} \\
  u &\mapsto \int_{\Omega \backslash \sigma} Ae(u) : e(u) \diff x + k\mathcal{H}^1(\sigma)\,,
  \end{align*}
  
  Une solution du problème de fracture fragile est un couple $(u^*, \sigma^*)$ qui vérifie:
  \vspace*{0.25cm}
  $$
  \etot(u^*,\sigma^*) = \min_{\sigma \in \Sigma}{\min_{u\in A_{\sigma}}{\etot(u,\sigma)}} \,.
  $$

  \myfig{PhaseField2.png}{Bifurcation d’une fracture}

\end{frame}


\begin{frame}{Réseaux de ressorts régulier}

  \myfigsize{Ressort1.png}{Réseau de ressorts régulier}{20}

  \vspace*{-0.25cm}
  \myfigsize{VoronoiDelaunay.png}{Tirage de points et diagrammes de Voronoi (à gauche) et Delaunay (à droite)}{60}

\end{frame}
