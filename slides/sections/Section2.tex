
%-------------------------------------------------------------------------------
%							SECOND SECTION
%-------------------------------------------------------------------------------

\section{\textsc{État de l'art}}


\subsection{Thèse de M. Rabatel}


\begin{frame}{Cinétique du floe}
  \mycols{
    \mycol{50}{
      Les équations de Newton-Euler:
      \begin{align} \label{eq:neweu}
      \begin{dcases} 
        M_i \frac{\diff \dot{\bvec{G}}_i(t)}{\diff t} &= \bvec{F}_i, \\
        \mathcal{I}_i \frac{\diff \dot{\theta}_i(t)}{\diff t} &= \mathfrak{M}_i,
      \end{dcases}        
      \end{align}
      où pour le floe $i$ : 
      \begin{itemize}
        \item $M_i$ : masse du floe;
        \item $\bvec{F}_i$ : somme des forces par unité de volume;
        \item $\mathcal{I}_i$ : le moment d'inertie du floe $i$;
        \item $\mathfrak{M}_i$ : le moment dynamique en $G$.
      \end{itemize}
      Le système (\ref{eq:neweu}) se réécrit sous la forme :
      \begin{align*}    
          \mathcal{M}_i \frac{\diff W_i(t)}{\diff t} = \mathcal{H}_i(t) ,
      \end{align*}
      avec 
      \begin{align*}
      \mathcal{M}_i = 
      \begin{pmatrix}
          M_i & 0 & 0 \\ 0 & M_i & 0 \\ 0 & 0 & \mathcal{I}_i
      \end{pmatrix} , \quad
      W_i(t) = 
      \begin{pmatrix}
          \dot{\bvec{G}}(t) \\ \dot{\theta}_i(t)
      \end{pmatrix} , 
      \text{ et } \quad \mathcal{H}_i(t) = 
      \begin{pmatrix}
          \bvec{F}_i(t) \\ \mathfrak{M}_i(t)
      \end{pmatrix}.
      \end{align*}
    }


    \mycol{50}{
      \myfig{FloeRabatel}{Repères abosolu et repère local pour une particule floe}
    }
  }
  
\end{frame}



\begin{frame}{Résumé de la thèse}
  \begin{itemize}
    \item Les floes sont rigides;
    \item Les coefficients de restitution;
    \item \dots
  \end{itemize}
\end{frame}

