
%-------------------------------------------------------------------------------
%							THIRD SECTION
%-------------------------------------------------------------------------------



\section{\textsc{Results in 1D}}


\subsection{Simulation}


\begin{frame}[fragile]
  \frametitle{Example of a 1D simulation}

  \begin{center}
    \textcolor{violet}{\href{run:../img/Video1D.mp4}{Click here for the RTE simulation in 1D.}}
    % \href{run:./img/Video1D.mp4}{Click here for the RTE simulation in 1D.}
  \end{center}

% \begin{center}
%   \includemedia[
%     width=0.95\linewidth,
%     activate=pageopen,
%     addresource=Video1D.mp4,
%     flashvars={
%        source=Video1D.mp4
%       &autoPlay=true
%     },
%     passcontext, % enable VPlayer's right-click menue
%   ]{\includegraphics{Thumbnail1D.png}}{VPlayer.swf}%
% \end{center}

\end{frame}

\begin{frame}[fragile]
  \frametitle{Inputs/outputs in 1D}

  \begin{block}{\vspace*{-3cm}}
    \centering The data gathered from the simulations is post-processed (resampled, normalized, etc.). 
  \end{block}

  \begin{columns}
  \begin{column}{0.7\textwidth}
      \begin{figure}
      \includegraphics[width=8cm]{EntreeSortie1D}       
      \mycap{An input for the Neural Network (above) and the awaited output (below)}
      \end{figure}
   \end{column}
   \begin{column}{0.3\textwidth}
      \begin{figure}
      \includegraphics[width=3cm]{Entrees1D}       
      \mycap{Size of an input (from the right boundary)}
      \end{figure}
      \begin{figure}
        \includegraphics[width=2cm]{Sortie1D.png}       
        \mycap{Size of an output}
      \end{figure}
   \end{column}
  \end{columns}

\end{frame}




\subsection{Model, training, and predictions}



\begin{frame}
  \frametitle{The architecture in Keras}
  \begin{center}
      %%% Ajouter l'image des hierarchie de machine learning
      \begin{figure}
      \includegraphics[width=.95\textwidth]{DRNNNew.png}    % Modifier la sortie pour avoir 1D/Class2D/Reg 2D
      \mycap{CNN used for the 1D inverse problem}
      \end{figure}
  \end{center}
\end{frame}



\begin{frame}
  \frametitle{Hyper-parameters and training}

  \begin{table}[h!]
      \scriptsize
      \label{tab:Parametres}
      \centering
      \begin{tabular}{l l}
      \toprule
      \textbf{Hyper-parameter} & \hspace*{2mm}\textbf{Value} \\
      \midrule
      % \onslide<+>
      optimizer  & Adam \\
      learning rate  & 1e-4, 1e-5, etc. \\
      batch size  & 32 \\
      epochs  & 100 \\
      patience  & 10 \\
      kernel size  & 3 \\
      activation  & relu, linear, sigmoid \\
      \bottomrule
      \end{tabular}
      \mycaptab{List of the most important hyper-parameters for the training}
  \end{table}

  \myfig[7][train1D]{CourbeTraining1D.png}{$MSE$ loss and $R^2$ accuracy during training and validation}

\end{frame}


\begin{frame}[fragile]
  \frametitle{Best/worst predictions}

  \begin{columns}
  \begin{column}{0.5\textwidth}
      \begin{figure}
      \includegraphics<1->[width=4cm]{Meilleur1D2}       
      \includegraphics<1->[width=4cm]{Meilleur1D1}       
      \only<1->{\mycap{The 2 best predictions}}
      \end{figure}
   \end{column}
   \begin{column}{0.5\textwidth}
      \begin{figure}
      \includegraphics[width=4cm]{Pire1D1}       
      \includegraphics[width=4cm]{Pire1D2}       
      \only{\mycap{The 2 worst predictions}}
      \end{figure}
   \end{column}
  \end{columns}

\end{frame}


\begin{frame}
  \frametitle{The obtained scores}

  \begin{table}[h!] 
      \centering
      \begin{tabular}{l l}
      \toprule
      \textbf{Score name} & \textbf{Value} \\
      \midrule
      $R^2$ & 45.50 \%\\
      \textbf{Personalized} & 28.21 \%\\
      \bottomrule
      \end{tabular}
      \mycaptab{Prediction results on the Test dataset in 1D}
  \end{table}

  \begin{columns}
       \begin{column}{0.5\textwidth}
          \begin{figure}
          \includegraphics<1>[width=3cm]{Hauteur1D}       
          \only<1>{\mycap{Correlation between observed and predicted \alert{heights} of the obstacles}}
          \end{figure}
       \end{column}
       \begin{column}{0.5\textwidth}
        \begin{figure}
        \includegraphics<1->[width=3cm]{Position1D}       
        \only<1->{\mycap{Correlation between observed and predicted \alert{positions} of the obstacles}}
        \end{figure}
     \end{column}
  \end{columns}

\end{frame}


\metroset{background=dark}

\begin{frame}
  \frametitle{Conclusion on the 1D inverse problem}
  
  \begin{alertblock}{\textbf{What we found out in 1D is:}}
    \begin{enumerate}
      \item We are able to determine the size of the tumours (obstacles);
      \item However, we cannot locate them.
    \end{enumerate}
    The 1D inverse problem appears to be ill-posed. We could try solving the issue by increasing the dimensionality.
  \end{alertblock}
\end{frame}

\metroset{background=light}


