
%-------------------------------------------------------------------------------
%							FIRST SECTION
%-------------------------------------------------------------------------------

\section{\textsc{Introduction}}


\subsection{Test subsection title}


\begin{frame}[fragile,t]
  \frametitle{Motivation}
  
  \note{Floe: Un floe est un morceau de glace.}

  \mycols{

    \mycol{50}{
      \begin{exampleblock}{Enjeux écologiques}
        \begin{itemize}
          \item Etude climatique à échelle nature (SASIP)
          \item Prévisions climatiques avec précision
        \end{itemize}
      \end{exampleblock}

      \myfig{Ecological.jpg}{Prévision dans l'artique}
      }

    \pause

    \mycol{50}{
      \begin{alertblock}{Enjeux industrielles}
        \begin{itemize}
          \item Routes maritimes exploitables
          \item Comportemetn des stations offshores 
        \end{itemize}
      \end{alertblock}
      \myfig{IceRoutes.png}{Un navire dans la MIZ}

      }
  }
\end{frame}


% \section{\textsc{Introduction2}}

\begin{frame}{Objectifs}

  \begin{exampleblock}{Objectifs généraux}

    \begin{itemize}
      \item Modélisation et analyse mathématique de la notion de percussion
      \item Poursuite du développement d’un modèle de fracturation des floes
    \end{itemize}
  \end{exampleblock}


  \begin{block}{Objectifs intermédiaires}

    \begin{enumerate}
      \item Lecture des travaux précédents:
      \begin{itemize}
        \item M. Rabatel, S. Labbé, et J. Weiss: Dynamics of an assembly of rigid ice floes (\citeyear{rabatel2015dynamics}); 
        \item Matthias Rabatel: Modélisation dynamique d’un assemblage de floes rigides (\citeyear{rabatel2015modelisation});
        \item Dimitri Balasoiu: Modélisation et simulation du comportement mécanique de floes de glace (\citeyear{balasoiu2020modelisation}).
      \end{itemize}
      
      \item Modélisation et simulation du deplacmeent des noeuds d'un floe isolé:
      \begin{itemize}
        \item en 1D;
        \item en 2D.
      \end{itemize}

      \item Introduction de la percussion dans le code préexistant.

    \end{enumerate}
  \end{block}

\end{frame}
