
%-------------------------------------------------------------------------------
%							FIRST SECTION
%-------------------------------------------------------------------------------

\section{\textsc{Motivation}}




\begin{frame}[fragile,t]
  \small
  \frametitle{Motivation}
  
  \note{
    \begin{description}
      \item Floe: Un floe est un morceau de glace;
      \item Albédo: la cryosphère reflette entre 90 et 95 \% des rayonnement recus par la planète.
    \end{description} 
  }

  \vspace*{-0.4cm}
  \begin{columns}[onlytextwidth,t]

    \mycol{50}{
      \begin{exampleblock}{Enjeux écologiques}
        \begin{itemize}
          \item Rôle crucial de la zone polaire pour le climat face au réchauffement climatique (SASIP)
          \item Prévisions climatiques à échelle nature
        \end{itemize}
      \end{exampleblock}

      \vspace*{-0.1cm}
      \myfig{Ecological.jpg}{Image satellite de l'Arctique (MIZ Program, 2013)}
      }

    \pause

    \mycol{50}{
      \begin{alertblock}{Enjeux industriels}
        \begin{itemize}
          \item Ouverture des routes maritimes pour l'exploitation des hydrocarbures
          \item Étude de l'interaction stations offshores / glace 
        \end{itemize}
      \end{alertblock}

      \vspace*{-0.1cm}
      \myfig{IceRoutesSlides.png}{Un bateau industriel dans la MIZ (O Globo, 2012)}

      }

    \end{columns}

    \note{La MIZ c'est la Maginal Ice Zone c'est cette zone où la glace est disloquée avec environ 80 per cent. C'est une zone idéale pour appliquer un modèle granulaire (les grains étant les floes, ou les morceaux de glace).
    Pendant que tu parle de l'interaction offshore/glace, il faut se poser la question de savoir qu'est ce cède des deux (ca va introduire mes objectifs)}

    \pause
    \vspace*{-0.3cm}
    \begin{tcolorbox}[colback=red!5,colframe=red!30!black,arc=0mm]
      Il est donc urgent de prédire l’évolution de la banquise et de la MIZ (au moins) à court terme.
    \end{tcolorbox}
    \normalsize
\end{frame}


% \section{\textsc{Introduction2}}

\begin{frame}{Objectifs}

  \begin{exampleblock}{Objectifs généraux}

    \begin{itemize}
      \item Comprendre le modèle de rupture de Griffith \textcolor{red}{(6 semaines)}; 
      \item Comprendre le passage micro/macro de la percussion \textcolor{red}{(20 semaines)};
      \item Intégrer le modèle dans un code de calcul à l’échelle des floes de glace \textcolor{red}{(0 semaines)}.
    \end{itemize}
  \end{exampleblock}

  \pause

  \begin{block}{Objectifs intermédiaires}

    \begin{enumerate}
      \item Prise en main de la notion de $\Gamma$‑convergence;
      \item Lecture des travaux précédents:
      \begin{itemize}
        \item \alert{M. Rabatel, S. Labbé, et J. Weiss}: Dynamics of an assembly of rigid ice floes (\citeyear{rabatel2015dynamics}); 
        \item \alert{Matthias Rabatel}: Modélisation dynamique d’un assemblage de floes rigides (\citeyear{rabatel2015modelisation});
        \item \alert{Dimitri Balasoiu}: Modélisation et simulation du comportement mécanique de floes de glace (\citeyear{balasoiu2020modelisation}).
      \end{itemize}
      
      \item Modélisation et simulation du déplacement des n\oe{}uds d'un floe isolé:
      \begin{itemize}
        \item en 1D;
        \item en 2D;
      \end{itemize}

      \item Introduction de la percussion et de la fracture dans le code préexistant.

    \end{enumerate}
  \end{block}

\end{frame}
