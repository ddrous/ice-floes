
%-------------------------------------------------------------------------------
%							FIRST SECTION
%-------------------------------------------------------------------------------

\section{\textsc{Motivation}}




\begin{frame}[fragile,t]
  \frametitle{Motivation}
  
  \note{
    \begin{description}
      \item Floe: Un floe est un morceau de glace;
      \item Albédo: la cryosphère reflette entre 90 et 95 \% des rayonnement recus par la planète.
    \end{description} 
  }

  \vspace*{-0.3cm}
  \begin{columns}[onlytextwidth,t]

    \mycol{50}{
      \begin{exampleblock}{Enjeux écologiques}
        \begin{itemize}
          \item Etude climatique à échelle nature (SASIP)
          \item Prévisions climatiques avec précision
        \end{itemize}
      \end{exampleblock}

      \myfig{Ecological.jpg}{Image satellite de l'artique (MIZ Program, 2013)}
      }

    \pause

    \mycol{50}{
      \begin{alertblock}{Enjeux industriels}
        \begin{itemize}
          \item Routes maritimes exploitables
          \item Étude des stations offshores 
        \end{itemize}
      \end{alertblock}

      \myfig{IceRoutes.png}{Un bateau industriel dans la MIZ (O Globo, 2012)}

      }

    \end{columns}

\end{frame}


% \section{\textsc{Introduction2}}

\begin{frame}{Objectifs}

  \begin{exampleblock}{Objectifs généraux}

    \begin{itemize}
      \item Comprendre le modèle de rupture de Griffith \textcolor{blue}{(6 semaines)}; 
      \item Comprendre le passage micro/macro de la percussion \textcolor{blue}{(20 semaines)};
      \item Intégrer le modèle dans un code de calcul à l’échelle des floes de glace \textcolor{blue}{(0 semaines)}.
    \end{itemize}
  \end{exampleblock}


  \begin{block}{Objectifs intermédiaires}

    \begin{enumerate}
      \item Prise en main de la notion de $\Gamma$‑convergence;
      \item Lecture des travaux précédents:
      \begin{itemize}
        \item \alert{M. Rabatel, S. Labbé, et J. Weiss}: Dynamics of an assembly of rigid ice floes (\citeyear{rabatel2015dynamics}); 
        \item \alert{Matthias Rabatel}: Modélisation dynamique d’un assemblage de floes rigides (\citeyear{rabatel2015modelisation});
        \item \alert{Dimitri Balasoiu}: Modélisation et simulation du comportement mécanique de floes de glace (\citeyear{balasoiu2020modelisation}).
      \end{itemize}
      
      \item Modélisation et simulation du déplacement des n\oe{}uds d'un floe isolé:
      \begin{itemize}
        \item en 1D;
        \item en 2D;
      \end{itemize}

      \item Introduction de la percussion et de la fracture dans le code préexistant.

    \end{enumerate}
  \end{block}

\end{frame}
