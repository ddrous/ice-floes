
%-------------------------------------------------------------------------------
%							FIRST SECTION
%-------------------------------------------------------------------------------

\section{\textsc{Introduction}}


\subsection{Goals and incentives}


%-------- Vrai debut de l'introduction (PB INVERSE)
\begin{frame}
    \frametitle{Motivation }
  
    \begin{alertblock}{\textbf{Motivation for the project:}}      
      \begin{itemize}
        \item High demand for \textbf{Deep Learning} algorithms, with recent development in hardware and HPC; % Depuis le debut de la decenie 2010, le Machine Learning a considerablement pris de l’ampleur (2015 a l’ILSVRC, etc..)
        %%%%%% IMAGE DU DEEP LEARNING
        \item Re-evaluating how we solve \textbf{inverse problems} that often require complex optimization algorithms; % Les problemes inverses sont difficiles. ... Les algo d'optimisation classiques marchent tres bien. En fait on s'est referer aux travaux de Maya et Guillaume Dolle. L'avantage que peuvent offrir les ANN c'est juste la simplicite, et la rapidite, et une generalisation (non specificite aux probleme)
        %%%%%% IMAGE DU PB INVERSE
        \item Facilitate (early) detection of \textbf{cancer} by using AI for medical imaging. % Avant de soigner les cancers, on doit detecter les tumeurs sont plus denses que les tissus sains (Chercher d'autres applications)
        %%%%%% IMAGE DU MEDICAL
      \end{itemize}
    \end{alertblock}


    \begin{itemize}
      \item High demand %for \textbf{Deep Learning} algorithms, with recent development in hardware and HPC; % Depuis le debut de la decenie 2010, le Machine Learning a considerablement pris de l’ampleur (2015 a l’ILSVRC, etc..)
      %%%%%% IMAGE DU DEEP LEARNING
      \item Re-evaluating %how we solve \textbf{inverse problems} that often require complex optimization algorithms; % Les problemes inverses sont difficiles. ... Les algo d'optimisation classiques marchent tres bien. En fait on s'est referer aux travaux de Maya et Guillaume Dolle. L'avantage que peuvent offrir les ANN c'est juste la simplicite, et la rapidite, et une generalisation (non specificite aux probleme)
      %%%%%% IMAGE DU PB INVERSE
      \item Facilitate (early) % Avant de soigner les cancers, on doit detecter les tumeurs sont plus denses que les tissus sains (Chercher d'autres applications)
      %%%%%% IMAGE DU MEDICAL
    \end{itemize}
    % \begin{align*}
      
    % \alpha = \dfrac{1}{\Delta t \left( \frac{1}{\Delta t} + c \sigma_a \right)} ,\quad 
    %  \beta = \dfrac{c \sigma_a}{\frac{1}{\Delta t} + c \sigma_a} ,\quad 
    %  \\
    %  \gamma = \dfrac{\rho_j C_v \mu_q}{\Delta t \left( \frac{\rho_j C_v \mu_q}{\Delta t} + c \sigma_a \right)}, \quad \text{and} \quad  
    %  \delta = \dfrac{c \sigma_a}{\frac{\rho_j C_v \mu_q}{\Delta t} + c \sigma_a}.

    % \end{align*}


\end{frame}


  \subsection{Goals and incentives 2}


  \begin{frame}
    \frametitle{Motivation 2}
  
    \begin{exampleblock}{\textbf{Motivation for the project:}}      
      \begin{enumerate}
        \item High demand for \textbf{Deep Learning} algorithms, with recent development in hardware and HPC; % Depuis le debut de la decenie 2010, le Machine Learning a considerablement pris de l’ampleur (2015 a l’ILSVRC, etc..)
        %%%%%% IMAGE DU DEEP LEARNING
        \item Re-evaluating how we solve \textbf{inverse problems} that often require complex optimization algorithms; % Les problemes inverses sont difficiles. ... Les algo d'optimisation classiques marchent tres bien. En fait on s'est referer aux travaux de Maya et Guillaume Dolle. L'avantage que peuvent offrir les ANN c'est juste la simplicite, et la rapidite, et une generalisation (non specificite aux probleme)
        %%%%%% IMAGE DU PB INVERSE
        \item Facilitate (early) detection of \textbf{cancer} by using AI for medical imaging. % Avant de soigner les cancers, on doit detecter les tumeurs sont plus denses que les tissus sains (Chercher d'autres applications)
        %%%%%% IMAGE DU MEDICAL
      \end{enumerate}

      \begin{itemize}
        \item test 1
        \item test 2
      \end{itemize}
    \end{exampleblock}

    \begin{theorem}
      A is good
      \begin{equation}
        \begin{aligned}
        F ={} & \{F_{x} \in  F_{c} : (|S| > |C|) \\
              & \cap (\mathrm{minPixels}  < |S| < \mathrm{maxPixels}) \\
              & \cap (|S_{\mathrm{conected}}| > |S| - \epsilon)\}
        \end{aligned}
        \end{equation}
    \end{theorem}


  \end{frame}

