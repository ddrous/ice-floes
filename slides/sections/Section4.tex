
%-------------------------------------------------------------------------------
%							FOURTH SECTION
%-------------------------------------------------------------------------------


\section{\textsc{Results in 2D}}


\subsection{Simulation}

\setbeamercovered{invisible}

\begin{frame}
  \frametitle{The enriched problem in 2D}
  \begin{alertblock}{\vspace*{-3cm}}
    In 2D, we are not only studying the location and the height of an obstacle, we are trying to rebuild the complete density of the medium.     
  \end{alertblock}

  \begin{columns}
    \begin{column}{0.45\textwidth}
      \begin{center}
        %%% Ajouter l'image des hierarchie de machine learning
        \begin{figure}
        \includegraphics[width=.7\textwidth]{DensiteTest.png}    % Modifier la sortie pour avoir 1D/Class2D/Reg 2D
        \mycap{An example of a medium's density where we can see two distinct "tumours"}
        \end{figure}
    \end{center}
    \end{column}

    \pause

    \begin{column}{0.55\textwidth}
      \begin{center}
        %%% Ajouter l'image des hierarchie de machine learning
        \begin{figure}
        \includegraphics[width=.6\textwidth]{CTScan.jpg}    % Modifier la sortie pour avoir 1D/Class2D/Reg 2D
        \includegraphics[width=.5\textwidth]{Sinogram.png} 
        \mycap{CT scan overview (up), Shepp Logan Phantom (lower left), and corresponding sinogram (lower right) (Beatty, 2012)}
        \end{figure}
    \end{center}
    \end{column}

  \end{columns}
\end{frame}


\begin{frame}[fragile]
  \frametitle{Example of a 2D simulation}
  
  \begin{center}
    \textcolor{violet}{\href{run:../img/Video2D.mp4}{Click here for the RTE simulation in 2D.}}
    % \href{run:./img/Video2D.mp4}{Click here for the RTE simulation in 2D.}
  \end{center}

% \begin{center}
%   \includemedia[
%     width=0.95\linewidth,
%     activate=pageopen,
%     addresource=Video2D.mp4,
%     flashvars={
%        source=Video2D.mp4
%       &autoPlay=true
%     },
%     passcontext, % enable VPlayer's right-click menue
%   ]{\includegraphics{Thumbnail2D.png}}{VPlayer.swf}%
% \end{center}

\end{frame}

\begin{frame}[fragile]
  \frametitle{Inputs/outputs in 2D}

      \begin{figure}
      \includegraphics[width=11cm]{SimuCirculaires.png}       
      \mycap{An input for the Neural Network (around the pictures) and the awaited output (in the middles)}
      \end{figure}

      \begin{columns}
        \begin{column}{0.5\textwidth}
          \begin{figure}
            \includegraphics[width=3cm]{Entrees2D}       
            \mycap{Size of an input}
            \end{figure}
        \end{column}
        \begin{column}{0.5\textwidth}
          \begin{figure}
            \includegraphics[width=3cm]{Sortie2D.png}       
            \mycap{Size of an output}
            \end{figure}
        \end{column}
      \end{columns}


\end{frame}



\subsection{Model, training, and predictions}


\begin{frame}
  \frametitle{Model architecture in Keras and Tensorflow}
  \begin{center}
      %%% Ajouter l'image des hierarchie de machine learning
      \begin{figure}
      \includegraphics[width=.95\textwidth]{Vnet.png}    % Modifier la sortie pour avoir 1D/Class2D/Reg 2D
      \mycap{V-Net architecture \parencite{rapportstagevnet} based on U-Net \parencite{DBLP:journals/corr/RonnebergerFB15}}
      \end{figure}
  \end{center}
\end{frame}

\begin{frame}[fragile]
  \frametitle{Python code for the V-Net}
	\begin{lstlisting}[language=Python,caption={Definition of the constructor for the VNet class \parencite{rapportstagevnet}.},breaklines]
  class VNet:
    def __init__(self,
                  input_shape,
                  output_shape,                  
                  levels=5,
                  depth=32,
                  kernel_size=5, 
                  activation="relu",
                  batch_norm=True,
                  dropout_rate=0, 
                  ):
    ..."more code"
  \end{lstlisting}
\end{frame}

\begin{frame}
  \frametitle{Training phase}
  \begin{table}[h!]
      \scriptsize
      \centering
      \begin{tabular}{l l}
      \toprule
      \textbf{Hyper-parameter} & \hspace*{2mm}\textbf{Value} \\
      \midrule
      optimizer  & Adam\\
      learning rate  & 1e-5, 1e-6, \colorbox{yellow}{1e-7}\\
      batch size  & 100 \\
      epochs  & 1500 \\
      patience & 50 \\
      kernel size  & (5,5) \\
      activation & relu, linear, sigmoid \\
      \bottomrule
      \end{tabular}
      \mycaptab{List of the most important hyper-parameters for the training}
      \label{tab:Parametres}
  \end{table}

  \vspace*{-0.4cm}
  % \begin{center}
    %%% Ajouter l'image des hierarchie de machine learning
    \begin{figure}
    \includegraphics[width=.75\textwidth]{training1.png}    % Modifier la sortie pour avoir 1D/Class2D/Reg 2D
    \mycap{MSE drop (left) and accuracy increase (right) during training}
    \end{figure}
% \end{center}

\end{frame}



\begin{frame}[fragile]
  \frametitle{Best/worst predictions}

  \begin{block}{\vspace*{-3cm}}
      $R^2$ score of $87$ \%.
  \end{block}


  \begin{columns}
  \begin{column}{0.5\textwidth}
    \begin{figure}
      \includegraphics<1->[width=5cm]{Meilleur2D1}
    \end{figure}
    \begin{figure}
      \vspace*{-0.25cm}       
      \includegraphics<1->[width=5cm]{Meilleur2D2}       
      \only<1-> {\mycap{Some of the best predictions}}
    \end{figure}
   \end{column}
   \begin{column}{0.5\textwidth}
      \begin{figure}
        \includegraphics[width=5cm]{Pire2D1}     
      \end{figure}
      \begin{figure}
        \vspace*{-0.25cm}       
        \includegraphics[width=5cm]{Pire2D3}       
        \mycap{Some of the worst predictions}
      \end{figure}
   \end{column}
  \end{columns}

\end{frame}


\metroset{background=dark}

\begin{frame}
  \frametitle{Conclusion on the 2D inverse problem}
  \begin{alertblock}{\textbf{What we found out in 2D is:}}
    \begin{enumerate}
      \item The AI is able to determine the full density of the medium;
      \item The AI is more precise when the obstacles are continuous (gaussian, sine, etc.) as opposed to circles and rectangles.
    \end{enumerate}
    We can easily post-process the predicted density to find the tumours.
  \end{alertblock}

  \pause

  \begin{alertblock}{\textbf{Future work:}}
    While writing and publishing the present work, I plan on:
    \begin{enumerate}
      \item Reconstructing other optical properties ($\sigma_e$, $\sigma_a$, and $\sigma_c$);
      \item Finding out why \textbf{increasing} the dimensionality solves the problem (linked to the Radon Tansform);
      \item Exploring how a tomograph can be implemented and used in medical conditions.
    \end{enumerate}
  \end{alertblock}

\end{frame}

\metroset{background=light}

