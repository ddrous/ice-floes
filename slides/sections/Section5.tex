
%-------------------------------------------------------------------------------
%							FITH SECTION
%-------------------------------------------------------------------------------

\section{\textsc{Conclusion}}


\subsection{Apports et recherches ultérieures}

\begin{frame}{Apports et recherches ultérieures}

	\myfigframesize{Timeline}{Résumé du déroulement du stage}{60}
	
	\note{On a commencer par la lecture des travaux antérieus, et on a fini par la percussion 1D en pasant par la percussion 2D. Ca semble contre-intuitif MAIS c'est nécésssaire pour avancer.}
	
	\small
	\vspace{-0.4cm}
	\mycols{	
		
		\mycol{50}{
        \begin{exampleblock}{Apports du stage}
        \begin{itemize}
        \item Simulation de systèmes dynamiques en Python;
        \item Prise en main du modèle de rupture de Griffith (analyse fonctionnelle, analyse numérique, etc.);
        \item Maitrise de l'approche par réseaux de ressorts (probabilité, raisonnance);
        \item Utilisation de TikZ, Flask, Bokeh, Symbolab, et bien d'autres;
        \item Recherche en milieu professionnel;
        \item Savoir-faire transférables (vision globale, etc.).
        \end{itemize}
		\end{exampleblock}
		}
	
		\mycol{50}{
        \begin{alertblock}{Recherches ultérieures}
        \begin{itemize}
        \item Implémentation de la méthode du champ de phase;
        \item Implementation de la fracture au problème 2D, 2.5D, ou 3D;
        \item Intégration de la fracture fragile au code de \citeauthor{rabatel2015modelisation}
        \item Confirmation de l'approximation par réseaux de ressorts;
        \item Optimiser les codes avec Cython ou en C++;
        \item Tests de validation en laboratoire.
        \end{itemize}
		\end{alertblock}
		}
		
	}
    
\end{frame}




\subsection{Délivrables}

\begin{frame}{Checklist et délivrables}

        \begin{block}{Checklist des objectifs}
    	\begin{enumerate}
      	\item[\checkmark] Prise en main de la notion de $\Gamma$‑convergence;
      	\item[\checkmark] Assimilation des travaux antérieurs;
      	\item[\checkmark] Modélisation de la percussion (1D et 2D);
      	\item[\checkmark] Modélisation de la fracture:
      	\begin{itemize}
        \item[\checkmark] en 1D;
        \item[$\times$] en 2D.
        \end{itemize}
      	\item[$\times$] Calculs à l'échelle des floes de glace de l'Arctique.
      	\end{enumerate}
		\end{block}
	
        \begin{exampleblock}{Délivrables}
        \begin{enumerate}
        \item Rapport de stage: \myemoji \textcolor{orange}{\href{https://github.com/desmond-rn/ice-floes/tree/master/pdf}{ GitHub}};
        \item Code de calcul: \myemoji \textcolor{orange}{\href{https://github.com/desmond-rn/ice-floes/tree/master/code}{ GitHub}} et \myemoji \textcolor{orange}{\href{https://framagit.org/RaK/SimuRessorts}{ Framagit}};
        \item Quelques simulations: \myemoji \textcolor{orange}{\href{https://seafile.unistra.fr/d/a6c3680909624b22be7c/}{ Seafile}}.
        \end{enumerate}
		\end{exampleblock}
    
\end{frame}
