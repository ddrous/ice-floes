% Chapter 3

\chapter{Problème 1D et étude de la fracture} % 4th chapter title

\label{Chapter3} % For referencing the chapter elsewhere, use \ref{Chapter4} 

%1----------------------------------------------------------------------------------------









\section{Les différents modèles étudiés}









% \subsection{Collision parfaitement inélastique avec un floe encastré à l'instant initial}

% Nous effectuons ici une modélisation 1D de notre problème. Un floe est modélisé par un système masse-ressort de deux nœuds. Le floe 1 est immobilisé face au mur, et le floe 2 approche à la vitesse $\bvec{v}_0$. On identifie les nœuds $q_0$ et $p_0$ de la section précédente à leur masses respectives $m$ et $m'$ (voir \cref{fig:contact1d}).
% \begin{figure}[!h]
%     \centering
%     \frame{\includegraphics[width=0.8\textwidth]{Percussion1D-Systeme}}
%     \caption{Contact 1D parfaitement inélastique entre deux floes. Le floe percuté étant immobile et coincé au mur avant le choc.}
%     \label{fig:contact1d}
% \end{figure}

% \noindent On suppose que durant la dynamique non régulière, les masses $m$ et $m'$ en contact forment une seule masse\footnote{Cette simplification a pour principal avantage de supprimer le traitement de la force de contact entre les deux masses.}
% $m+m'$ dont
% le déplacement est donné par la variable $x_1(t)$. Le déplacement de la masse $m'$ à l'autre bout du floe percuteur est nommé
% $x_2(t)$. La masse $m$ qui est fixée au mur ne sera pas étudiée ici. Nous faisons à présent le bilan des forces qui
% s'exercent ces
% deux masses.
% \begin{figure}[!h]
%      \begin{subfigure}[b]{0.4\textwidth}
%          \centering
%          \frame{\includegraphics[width=\textwidth]{Percussion1D-Masse1}}
%          \caption{Sur $m+m'$.}
%          \label{fig:bilan11}
%      \end{subfigure}
% %     \hfill
%      \begin{subfigure}[b]{0.3\textwidth}
%          \centering
%          \frame{\includegraphics[width=\textwidth]{Percussion1D-Masse2}}
%          \caption{Sur $m'$.}
%          \label{fig:bilan12}
%      \end{subfigure}
%         \caption{Bilan des forces appliquée sur les noeuds du système. Les valeurs indiquées sont les intensitées
%             (positives) des forces durant une phase imaginée de compression des ressorts ($\bvec{v}_0 <0$ et donc
%             $x_1 <0$). Pour obtenir l'intesité de la force de rappel du ressort $k'$, on peut imaginer $x_1$ imobile
%             (on aura $x_2 < 0$, d'où $x_1 - x_2 > 0$) (voir \parencite{homodeling}).}
%         \label{fig:bilan}
% \end{figure}

% \noindent En orientant convenablement le système (voir \cref{fig:contact1d}), on applique la loi de Newton-Euler linéaire
% pour obtenir le système suivant et ses conditions initiales \footnote{J'ai des doutes sur cette condition
% initiale. La vitesse initiale de $x_1$ est-elle vraiment nulle?}:
% \begin{align}
%     \begin{dcases}
%     (m+m')\ddot x_1 = -kx_1 - \mu \dot x_1 + k'(x_2 - x_1) + \mu'(\dot x_2 - \dot x_1) \\
%         m' \ddot x_2 =  -k'(x_2 - x_1) - \mu'(\dot x_2 - \dot x_1) 
%     \end{dcases}
% \end{align}
% À l'instant initial $t_0$, on a le système suivant
% \begin{align} \label{eq:percussion1d}
%     \begin{dcases}
%     (x_1(t_0), x_2(t_0)) = (0,0) \\
%     (\dot x_1(t_0), \dot x_2(t_0)) = (0,-v_0)
%     \end{dcases}
% \end{align}
% En posant $X = (x_1, x_2)^T \in \mathbb{R}^2$, l' \cref{eq:percussion1d} devient
% \begin{align}
%     \underbrace{\mymat{m+m'}{0}{0}{m'}}_{A} \myvec{\ddot{x}_1}{\ddot{x}_2} = \underbrace{\mymat{-\mu -
%     \mu^\prime}{\mu'}{\mu'}{-\mu'}}_{B}
%     \myvec{\dot{x}_1}{\dot{x}_2} + \underbrace{\mymat{-k-k'}{k'}{k'}{-k'}}_{C} \myvec{x_1}{x_2} \,.
% \end{align}
% Puisque $m, m'\neq 0$, la matrice $A$ est inversible et on obtient au final le problème de Cauchy suivant:
% \begin{align} \label{eq:percussion1d2}
%     \begin{dcases}
%         \ddot{X}(t) = B' \dot{X}(t) + C'X(t) \,, \\
%         (X(t_0), \dot X (t_0)) = \left( \myvec{0}{0}, \myvec{0}{-v_0} \right) \,,
%     \end{dcases}
% \end{align}
% avec $B' = A^{-1}B$ et $C' = A^{-1}C$.

% \noindent Il s'agit la d'un système d'EDO du deuxième ordre à coefficients constants. Transformons le en un système du premier ordre pour
% une résolution plus aisée. On pose donc $Y= (X, \dot X)^T = (x_1, x_2, \dot{x}_1, \dot{x}_2)^T \in \mathbb{R}^4$ et le
% système
% \ref{eq:percussion1d2}
% devient
% \begin{align} \label{eq:systeme1d}
%     \begin{dcases}
%         \dot{Y}(t)= E Y(t) \\
%         Y_0 = Y(t_0) = (0,0,0,-v_0)^T
%     \end{dcases}
% \end{align}
% avec la matrice par blocs \[ E = \mymat{0}{I_2}{C'}{B'} \,, \] où $I_2$ désigne la matrice identité de
% $\mathbb{R}^{2\times2}$.

% \noindent Avec $t_0= 0$, la solution de ce système d'EDO du premier ordre à coefficients constants est unique et est donnée par
% \begin{align}
%     Y(t) = \exp(tE)Y_0
% \end{align}

% La résolution analytique du système passe par le calcul de l'exponentielle de la matrice $E \in \mathbb{R}^4$, ce qui s'avère difficile du à la taille de ladite matrice. Nous optons donc pour une solution numérique (voir figure \cref{fig:simucontact1d} issue du notebook $\verb|code/simu1D/Percussion1D-1.ipynb|$ ) \ldots
% \begin{figure}[!h]
%     \centering
%     \includegraphics[width=0.4\textwidth]{SimuPercussion1D.png}
%     \caption{Simulation de la percussion 1D entre deux floes avec $m=1$, $m'=1$, $k=16$, $k'=5$, $\mu=6$,
%         $\mu'=2$, $v_0=-1.0$, $t_{f}=32$. On observe effectivement le ralentissement du système et une convergence
%         vers l'état d'équilibre $Y_{eq}= (0,0,0,0)$.}
%     \label{fig:simucontact1d}
% \end{figure}

% \noindent Pour certaines valeurs (specifiquement de $\mu$ et $\mu'$), on constate que le système converge vers son état d'équilibre attendu $Y_{eq} = (0,0,0,0)$. Il nous reste dans cette section:
% \begin{enumerate}
%     \item Calculer analytiquement et numériquement tous les état d'équilibres $Y_{eq} \in \ker(E)$; distinguer les états stables des autres.
%     \item Calculer analytiquement l'exponentielle de la matrice $E$, et donner l'expression de la solution; déduire la condition sur les parametres pour que le système converge vers l'état d'équilibre voulu.
% \end{enumerate} 








% \subsubsection{Collision parfaitement inélastique sans présence du mur}

% Contrairement au cas étudié dans la section précédente, le mur est supprimé dans cette section. On obtient donc une troisième variable $x_3$ décrivant le comportement du noeud qui était rattaché au mur. La schéma régissant ce système est donnée à la \cref{fig:contact1d2}. Le bilan des forces appliquées aux noeuds est présenté à la \cref{fig:bilan2}.

% \begin{figure}[!h]
%     \centering
%     \frame{\includegraphics[width=0.8\textwidth]{Percussion1D-Systeme-2}}
%     \caption{Contact 1D parfaitement inélastique entre deux floes. Le floe percuté étant non immobile (et non coincé au mur) avant le choc. On représnte également les variables $x_1$, $x_2$, et $x_3$ décrivant les movements de chaque noeud.}
%     \label{fig:contact1d2}
% \end{figure}

% \begin{figure}[!h]
%     \begin{subfigure}[b]{0.25\textwidth}
%         \centering
%         \frame{\includegraphics[width=\textwidth]{Percussion1D2-Masse1.png}}
%         \caption{Sur $m$.}
%         \label{fig:bilan12}
%     \end{subfigure}
% %     \hfill
%     \begin{subfigure}[b]{0.31\textwidth}
%         \centering
%         \frame{\includegraphics[width=\textwidth]{Percussion1D2-Masse2}}
%         \caption{Sur $m+m'$.}
%         \label{fig:bilan22}
%     \end{subfigure}
% %     \hfill
%     \begin{subfigure}[b]{0.23\textwidth}
%         \centering
%         \frame{\includegraphics[width=\textwidth]{Percussion1D2-Masse3}}
%         \caption{Sur $m'$.}
%         \label{fig:bilan32}
%     \end{subfigure}
%        \caption{Bilan des forces appliquée sur les noeuds du système. On procède de facon similaire à \cref{fig:bilan} pour obtenir les sens et les intensités de ces forces.}
%        \label{fig:bilan2}
% \end{figure}

% \noindent Comme précédement, nous appliqons les lois de Newton pour obtenir:
% \begin{align}
%     \begin{dcases}
%     m\ddot x_1 = -k(x_1 - x_2) - \mu (\dot x_1 - \dot x_2) \,, \\
%     (m+m')\ddot x_2 = k(x_1 - x_2) + \mu (\dot x_1 - \dot x_2) - k'(x_2 - x_3) - \mu'(\dot x_2 - \dot x_3) \,, \\
%         m' \ddot x_3 =  k'(x_2 - x_3) + \mu'(\dot x_2 - \dot x_3) \,. 
%     \end{dcases}
% \end{align}
% Sous forme matricielle, on a
% \begin{align}
%     \underbrace{\mybigmat{m}{0}{0}{0}{m+m'}{0}{0}{0}{m'}}_{A} \mybigvec{\ddot{x}_1}{\ddot{x}_2}{\ddot{x}_3} =  
%     \underbrace{\mybigmat{-k}{k}{0}{k}{-k-k'}{k}{0}{k'}{-k'}}_{B} \mybigvec{x_1}{x_2}{x_3} + 
%     \underbrace{\mybigmat{-\mu}{\mu}{0}{\mu}{-\mu-\mu'}{\mu'}{0}{\mu'}{-\mu'}}_{C} \mybigvec{\dot{x}_1}{\dot{x}_2}{\dot{x}_3} \,.
% \end{align}
% Puisque $m, m'\neq 0$, la matrice $A$ est inversible. En posant $X = (x_1, x_2, x_3)^T \in \mathbb{R}^3$, le système d'EDO revient à l' \cref{eq:percussion1d22} suivante:
% \begin{align} \label{eq:percussion1d22}
%         \ddot{X}(t) = B' X(t) + C'\dot{X}(t) \,, 
% \end{align}
% où $B' = A^{-1}B$ et $C' = A^{-1}C$. On pose ensuite $Y= (X, \dot X)^T \in \mathbb{R}^6$ et le système \cref{eq:percussion1d22} devient 
% \begin{align} \label{eq:systeme1d2}
%         \dot{Y}(t)= E Y(t)
% \end{align}
% avec $$ E = \mymat{0}{I_3}{B'}{C'} \,.$$


% Remarquons qu'en enlevant le mur à gauche du domaine (voir \cref{fig:contact1d}), le système est devenu isolé. Nous pouvons donc appliquer la conservation de la quantité de mouvement pour identifier la vitesse de l'ensemble $m+m'$ après collision et fixation de la masse $m'$ (à vitesse $\bvec{v}_0$) sur la masse $m$ (de vitesse $\bvec{v}'_0$)\footnote{Le vecteur $\bvec{v}'_0$ n'est pas marqué à la \cref{fig:contact1d2} (i.e. $\bvec{v}'_0 = 0$). L'introduction de ce vecteur permet de généraliser le problème.}. Pour simplifier les calculs, nous considérons les floes comme des solides rigides. La vitesse de l'ensemble juste après collision est notée $v_f$, et les quantités de mouvement avant et après choc sont notées $P_{\text{avant}}$ et $P_{\text{après}}$. On a :
% \begin{align*}
%     & \quad P_{\text{avant}} = P_{\text{après}} \\
%     \Rightarrow & \quad 2m \bvec{v}_0 + 2m'\bvec{v'}_0 = (2m + 2m') \bvec{v}_f \\
%     \Rightarrow & \quad \bvec{v}_f = \frac{m \bvec{v}_0 + m'\bvec{v'}_0}{m+m'}
% \end{align*} 

% \noindent On introduit ces conditions initiales dans l'\cref{eq:systeme1d2} pour obtenir le système de Cauchy ci-bas. Le résulat de la simulation est présenté à la figure \cref{fig:simucontact1d2} (issue du notebook $\verb|code/simu1D/Percussion1D-2.ipynb|$). 
% \begin{align} \label{eq:systeme1d3}
%     \begin{dcases}
%         \dot{Y}(t)= E Y(t) \,, \\
%         Y(t_0) = Y_0 = -v_f(0,0,0,1,1,1) \,.        
%     \end{dcases}
% \end{align}

% \begin{figure}[!h]
%     \centering
%     \begin{subfigure}{0.45\textwidth}
%         \centering
%         \includegraphics[width=\textwidth]{SimuPercussion1D2.png}
%         \caption{$k=3$}
%     \end{subfigure}
%     \begin{subfigure}{0.45\textwidth}
%         \centering
%         \includegraphics[width=\textwidth]{SimuPercussion1D2NonConv.png}
%         \caption{$k=23$}
%     \end{subfigure}

%     \caption{Simulation de la percussion 1D entre deux floes (sans présence du mur) avec $m=1$, $m'=1$, $k'=22$, $\mu=6$, $\mu'=2$, $v_0=-1.8$, $t_{f}=5$. Sous certaines conditions (forte dissipation, raideur du floe percuté élevée, etc.), on observe le ralentissement du système et une convergence vers l'état d'équilibre $Y_{eq}=(0,0,0,0,0,0)$.} 
%     \label{fig:simucontact1d2}
% \end{figure}

% \noindent La \cref{fig:simucontact1d2} permet d'observer la nuance avec le problème de contact parfaitemetn inélastique. Il est difficile de distinguer les cas qui aboutissent à une convergences des déplacements de ceux qui divergent. Observons donc à présent un problème de contact inélastique avec séparation des masses.





% \subsubsection{Modélisation du déplacement d'un floe isolé}
% \label{subsubsec:moddep1D}


% Avant d'entamer la question de la percussion avec séparation des masses (voir \cref{subsubsec:colinesepma}), étudions le comportement d'un floe de glace 1D isolé et modélisé par un réseau de ressorts (1 ressort, 1 dispositif visqueu, et 2 noeuds) (voir \cref{fig:deplacement1d}).
% \begin{figure}[!h]
%     \centering
%     \frame{\includegraphics[width=0.6\textwidth]{Deplacement1D-Systeme.png}}
%     \caption{Floe de glace 1D modélisé par un réseau de ressorts. Le floe est isolé de toutes forces extérieurs. Les varaibles $x_1$ et $x_2$ traduisent les déplacemnts des noeuds de gauche et de droite respectifs. À l'instant initial, les masses sont soumises aux vitesses $v_0$ et $v'_0$ indiquées.}
%     \label{fig:deplacement1d}
% \end{figure}


% \begin{figure}[!h]
%     \begin{subfigure}[b]{0.33\textwidth}
%         \centering
%         \frame{\includegraphics[width=\textwidth]{Deplacement1D-Masse1.png}}
%         \caption{Sur la masse $m$ de gauche.}
%     \end{subfigure}
% %     \hfill
%     \begin{subfigure}[b]{0.3\textwidth}
%         \centering
%         \frame{\includegraphics[width=\textwidth]{Deplacement1D-Masse2.png}}
%         \caption{Sur la masse $m$ de droite.}
%     \end{subfigure}
%        \caption{Bilan des forces appliquée sur les noeuds du système. Les valeurs indiquées sont les intensitées (positives) des forces (par exemple juste après l'instant initial, on a $x_1 > 0$, et $x_2 < 0$ d'où $k(x_1-x_2) > 0$).}
%        \label{fig:bilan0}
% \end{figure}


% \noindent Un bilan des forces effectué sur les deux noeuds du floe (voir \cref{fig:bilan0}) permet d'obtenir les équations suivantes:
% \begin{align}
%     \begin{dcases}
%     m\ddot x_1 = - k(x_1 - x_2) - \mu(\dot x_1 - \dot x_2) \,,\\
%         m \ddot x_2 =  k(x_1 - x_2) + \mu(\dot x_1 - \dot x_2) \,. 
%     \end{dcases}
% \end{align}
% En remarquant que $m\neq 0$, on passe à la forme matricielle qui s'écrit:
% \begin{align}
%     \myvec{\ddot{x}_1}{\ddot{x}_2} = 
%       \underbrace{ \frac{k}{m} \mymat{-1}{1}{1}{-1}}_{B} \myvec{x_1}{x_2}
%     + \underbrace{\frac{\mu}{m} \mymat{-1}{1}{1}{-1}}_{C} \myvec{\dot{x}_1}{\dot{x}_2} \,.
% \end{align}
% On pose ensuite la matrice par blocs:
% \[ E = \mymat{0}{I_2}{B}{C}  =  \begin{pmatrix}
%     0 & 0 & 1 & 0 \\ 0 & 0& 0& 1 \\ -\frac{k}{m} & \frac{k}{m} & -\frac{\mu}{m} & \frac{\mu}{m} \\ \frac{k}{m} & -\frac{k}{m} & \frac{\mu}{m} & -\frac{\mu}{m}
% \end{pmatrix}   \in \mathbb{R}^{4\times 4} \,, \quad \text{où} \quad I_2 = \mymat{1}{0}{0}{1} \in \mathbb{R}^{2\times 2} \,. \]
% On pose maintenant $Y = (x_1, x_2, \dot{x}_1, \dot{x}_2) \in \mathbb{R}^4$, et on reprend la condition initiale pour obtenir le système de Cauchy:
% \begin{align} \label{eq:dep1d}
%     \begin{dcases}
%         \dot{Y}(t)= E Y(t) \,, \\
%         Y_0 = Y(t_0) = (0,0,v_0,-v'_0)^T \,.
%     \end{dcases}
% \end{align}

% La solution numérique est présentée dans à la \cref{fig:simudept1d} (voir fichier \verb|code/simu1D/Deplacement1D-1.ipynb| pour plus de détails). La plus grosse remarque à faire du point de vue numérique est que lorsque $v_0 \neq v'_0$, les vitesses convergent vers $0$, mais les déplacements diverge.
% \begin{figure*}[!h]
%     \centering

%     \begin{subfigure}[t]{0.45\textwidth}
%         \centering
%         \includegraphics[width=\textwidth]{SimuDeplacement1D1.png}
%         \caption{$v_0=v'_0 = 0.8$}
%     \end{subfigure}
%     \begin{subfigure}[t]{0.45\textwidth}
%         \centering
%         \includegraphics[width=\textwidth]{SimuDeplacement1D2.png}
%         \caption{$v_0= 0.6$ et $v'_0 = 0.8$}
%     \end{subfigure}

%     \caption{Simulation du déplacement 1D d'un floe avec $m=1$, $k=18$, $\mu=1.3$, $t_{f}=5$. En règle générale, on observe le ralentissement du système et une convergence des déplacements vers l'état d'équilibre $Y_{eq}= (0,0,0,0)$ lorsque $v_0 = v'_0$.}
%     \label{fig:simudept1d}
% \end{figure*}

% Avec $t_0= 0$, la solution analytique de ce système d'EDO du premier ordre à coefficients constants est unique et est donnée par.
% \begin{align}
%     Y(t) = \exp(tE)Y_0 \,.
% \end{align}
% Nous obtenons le théorème suivant:
% \begin{theorem}[Convergence du modèle 1D isolé] \label{th:div1D}
%     Les déplacements $x_1$ et $x_2$ des noeuds du floe 1D convergent si et seulement si leurs vitesses initiales sont des vecteurs opposés.
% \end{theorem}

% \begin{proof}

% Le calcul des solution analytique est plus délicat. Il faudrait calculer l'exponentielle de la matrice $E$. Pour celà, nous devons diagonaliser (ou du moins trogonaliser) la matrice $E$. Son polynome caractéristique est donné par:
% \begin{align*}    
% \text{det}(E-\lambda I_4) &&&= \begin{vmatrix}
%     -\lambda & 0 & 1 & 0 \\ 0 & -\lambda & 0& 1 \\ -\frac{k}{m} & \frac{k}{m} & -\frac{\mu}{m}-\lambda & \frac{\mu}{m} \\ \frac{k}{m} & -\frac{k}{m} & \frac{\mu}{m} & -\frac{\mu}{m} -\lambda
% \end{vmatrix}, \\
%     &&&= \frac{\lambda^2}{m} \left( m\lambda^2 + 2\mu\lambda + 2k \right).
% \end{align*}
% Posons $\Delta = 4\mu^2 - 8km$. On distingue deux cas:
% \begin{itemize}
%     \item Si $\Delta \geq 0$: on pose $\lambda_1 = \frac{-\mu - \sqrt{\mu^2 - 2km}}{m}$ et $\lambda_2 = \frac{-\mu + \sqrt{\mu^2 - 2km}}{m}$;
%     \item Si $\Delta < 0$: on pose $\lambda_1 = \frac{-\mu - i\sqrt{2km - \mu^2}}{m}$ et $\lambda_2 = \frac{-\mu + i\sqrt{2km - \mu^2}}{m}$.
% \end{itemize}
% Nous avons donc exhiber les trois valeurs propres de notre matrice: $\lambda_0 = 0$, $\lambda_1$, et $\lambda_2$. Avec $\lambda$ désignant l'une des valeurs propres, on recherche les vecteurs $x = (x_1, x_2, x_3, x_4)^T \in \mathbb{R}^4$ appartenant aux sous espaces propres $E_\lambda$. On a:
% \begin{align} \label{eq:espacepropre}
%     Ex = \lambda x \Rightarrow \begin{dcases}
%         x_3 = \lambda x_1 \\
%         x_4 = \lambda x_2 \\
%         -(k + \mu \lambda + m \lambda^2) x_1 + (k + \mu \lambda) x_2 = 0 \\
%         (k + \mu \lambda) x_1 - (k + \mu \lambda + m \lambda^2) x_2 = 0
%     \end{dcases}
% \end{align}
% \begin{itemize}
%     \item Pour $\lambda = 0$, l'\cref{eq:espacepropre} revient à:
%     \begin{align*}
%         \begin{dcases}            
%         x_3 = 0 \\
%         x_4 = 0 \\
%         x_1 - x_2 = 0
%         \end{dcases}
%     \end{align*} 
%     On en déduit $E_0 = \text{vect}\{ e_1 \}$, avec $e_1 = (1,1,0,0)^T$.
%     \item Pour $\lambda = \lambda_1, \lambda_2$, on remarque que $k + \mu \lambda + m \lambda^2 = - (k + \mu \lambda)$. l'\cref{eq:espacepropre} revient donc à:
%     \begin{align*}
%         \begin{dcases}            
%         x_3 = \lambda x_1 \\
%         x_4 = \lambda x_2 \\
%         x_1 + x_2 = 0
%         \end{dcases}
%     \end{align*} 
%     On en déduit donc $E_{\lambda_1} = \text{vect}\{ e_3 \}$, avec $e_3 = (1,-1,\lambda_1,-\lambda_1)^T$; et $E_{\lambda_2} = \text{vect}\{ e_4 \}$ avec $e_4 = (1,-1,\lambda_2,-\lambda_2)^T$.
% \end{itemize}
% La meutilisicté arithmetique de $\lambda = 0$ est differente de sa multiplicité géometrique. La matrice $E$ n'est donc pas diagonalisable. Son polynome caractéristique étant scindé, nous alons la trigonaliser. On pose donc une base $\mathcal{B}' = (v_1, v_2, v_3, v_4)$ dans laquelle la matrice $E$ s'exprime par:
% \begin{align*}
%     P^{-1}EP = \begin{pmatrix}
%         0 & a & b & c \\ 0 & 0 & d & e \\ 0 & 0 & \lambda_1 & f \\ 0 & 0 & 0 & \lambda_2
%     \end{pmatrix},
% \end{align*}
% où $P$ est la matrice de passage de la base canonique de $\mathbb{R}^4$ (notée $\mathcal{B}$) à $\mathcal{B}'$. On a:
% \begin{itemize}
%     \item Dans $\mathcal{B}'$, le vecteur $v_1$ s'écrit $v_1 = (1,0,0,0)^T$ et on a $P^{-1}EP v_1 = 0$. $v_1$ est donc le vecteur propre associé à $0$ et on prend $v_1 = e_1 = (1,1,0,0)^T$ dans $\mathcal{B}$;
%     \item Dans $\mathcal{B}'$, le vecteur $v_2$ s'écrit $v_2 = (0,1,0,0)^T$ et on a $P^{-1}EP v_2 = a v_1$. On retourne dans $\mathcal{B}$ en posant $v_2 = (x_1, x_2, x_3, x_4)^T$ pour obtenir le système:
%     \begin{align*}
%         E v_2 = a v_1 \Rightarrow
%         \begin{dcases}            
%         x_3 = a  \\
%         x_4 = a  \\
%         x_1 - x_2 = 0
%         \end{dcases}.
%     \end{align*} 
%     Avec $a = 1$, on écrit $v_2 = e_2 = (1,1,1,1)^T$.
%     \item Dans $\mathcal{B}'$, le vecteur $v_3$ s'écrit $v_1 = (0,0,1,0)^T$ et on a $P^{-1}EP v_1 = \lambda_1 v_1 + bv_1 + d v_2$. En posant $b=d=0$, $v_1$ devient un vecteur propre associé à $\lambda_1$ et on prend $v_3 = e_3 = (1,-1,\lambda_1,-\lambda_1)^T$ dans $\mathcal{B}$;
%     \item De facon similaire, on obtient $v_4 = e_4 = (1,-1,\lambda_2,-\lambda_2)^T$ en posant $c=e=f=0$.
% \end{itemize}
% Nous avons donc trigonaliser la matrice $E$, et on écrit :
% \begin{align*}
%     P^{-1}EP = A, \text{avec} A = \begin{pmatrix}
%         0 & 1 & 0 & 0 \\ 0 & 0 & 0 & 0 \\ 0 & 0 & \lambda_1 & 0 \\ 0 & 0 & 0 & \lambda_2
%     \end{pmatrix}, \text{  } P = \begin{pmatrix}
%         1 & 1 & 1 & 1 \\ 1 & 1 & -1 & -1 \\ 0 & 1 & \lambda_1 & \lambda_2 \\ 0 & 1 & -\lambda_1 & -\lambda_2
%     \end{pmatrix}, \text{ et } P^{-1} = \frac{1}{2}\begin{pmatrix}
%         1 & 1 & -1 & -1 \\ 0 & 0 & 1 & 1 \\ \frac{\lambda_2}{\lambda_2-\lambda_1} & -\frac{\lambda_2}{\lambda_2-\lambda_1} & -\frac{1}{\lambda_2-\lambda_1} & \frac{1}{\lambda_2-\lambda_1} \\ -\frac{\lambda_1}{\lambda_2-\lambda_1} & \frac{\lambda_1}{\lambda_2-\lambda_1} & \frac{1}{\lambda_2-\lambda_1} & -\frac{1}{\lambda_2-\lambda_1}
%     \end{pmatrix}.
% \end{align*}
% La matrice $A$ se décompose en somme d'une matrice diagonale et d'une matrice nilpotente $A = D+N$ avec:
% \begin{align*}
%     D = \begin{pmatrix}
%         0 & 0 & 0 & 0 \\ 0 & 0 & 0 & 0 \\ 0 & 0 & \lambda_1 & 0 \\ 0 & 0 & 0 & \lambda_2
%     \end{pmatrix}, \text{ et } N = \begin{pmatrix}
%         0 & 1 & 0 & 0 \\ 0 & 0 & 0 & 0 \\ 0 & 0 & 0 & 0 \\ 0 & 0 & 0 & 0
%     \end{pmatrix}.
% \end{align*}
% En posant $E = P(D+N)P^{-1}$, nous pouvons facilemtn calculer $\forall t \in \mathbb{R}$, $\exp(tE) = P\exp(tD)\exp(tN)P^{-1}$. Ce calcul délicat donne (à l'aide du logiciel de calcul symbolique $\verb|Symbolab|$):
% \begin{align*}
%     \exp(tE) = \tiny \frac{1}{2(\lambda_2 - \lambda_1)}\begin{pmatrix} 
%         \lambda_2e^{t\lambda_1} + \lambda_2 - \lambda_1 - \lambda_1e^{t\lambda_2} & -\lambda_2e^{t\lambda_1} + \lambda_2 - \lambda_1 + \lambda_1e^{t\lambda_2} & t(\lambda_2 -\lambda_1) - e^{t\lambda_1} + e^{t\lambda_2} & t(\lambda_2 -\lambda_1) + e^{t\lambda_1} - e^{t\lambda_2} \\
%          -\lambda_2e^{t\lambda_1} + \lambda_2 - \lambda_1 + \lambda_1e^{t\lambda_2} & \lambda_2e^{t\lambda_1} + \lambda_2 - \lambda_1 - \lambda_1e^{t\lambda_2} & t(\lambda_2 -\lambda_1) + e^{t\lambda_1} - e^{t\lambda_2} & t(\lambda_2 -\lambda_1) - e^{t\lambda_1} + e^{t\lambda_2} \\
%           \lambda_1\lambda_2 (e^{t\lambda_1} - e^{t\lambda_2}) & \lambda_1\lambda_2 (e^{t\lambda_2} - e^{t\lambda_1})  & -\lambda_1e^{t\lambda_1} + \lambda_2 - \lambda_1 + \lambda_2e^{t\lambda_2} & \lambda_1e^{t\lambda_1} + \lambda_2 - \lambda_1 - \lambda_2e^{t\lambda_2} \\
%           \lambda_1\lambda_2 (e^{t\lambda_2} - e^{t\lambda_1})  & \lambda_1\lambda_2 (e^{t\lambda_1} - e^{t\lambda_2})  & \lambda_1e^{t\lambda_1} + \lambda_2 - \lambda_1 - \lambda_2e^{t\lambda_2} & -\lambda_1e^{t\lambda_1} + \lambda_2 - \lambda_1 + \lambda_2e^{t\lambda_2}
%     \end{pmatrix}.
% \end{align*}
% Rappelons nous que la solution du problème de Cauchy \cref{eq:dep1d} est donnée par $Y(t) = \exp(tE)Y_0$, avec $Y_0 = (0,0,v_0,-v'_0)$. Le calcul du déplacement $x_1$ donne:
% \begin{align} \label{eq:solreel}
%     x_1(t) = \frac{t}{2}\left( v_0 - v'_0 \right) - \frac{e^{t\lambda_1} - e^{t\lambda_2}}{2(\lambda_2 - \lambda_1)}\left( v_0 + v'_0 \right).
% \end{align}
% Le cas où $\Delta < 0$ (à étudier dans $\mathbb{C}$) peut se ramener au cas réel (dans $\mathbb{R}$) en posant $\lambda_1 = \alpha + i \beta$ et $\lambda_2 = \alpha - i \beta = \bar{\lambda}_1$ (avec $\alpha = -\frac{\mu}{m}$ et $\beta = -\frac{\sqrt{2km - \mu^2}}{m}$). En remarquant que $\sin(\beta t) = \frac{e^{i\beta t} - e^{-i\beta t}}{2i}$, on obtient :
% \begin{align} \label{eq:solcomplexe}
%     x_1(t) = \frac{t}{2}\left( v_0 - v'_0 \right) + \frac{e^{\alpha t} \sin(\beta t)}{2\beta} \left( v_0 + v'_0 \right).
% \end{align}
% Les \cref{eq:solreel,eq:solcomplexe} permettent d'observer que le déplacement $x_1$ ne converge pas lorsque $t \rightarrow +\infty$, à moins que $v_0 = v'_0$, ce qui est observé à la \cref{fig:simudept1d}. Pour le déplacement du deuxième noeud, on a:
% \begin{align} \label{eq:solcomplexe}
%     x_2(t) = \frac{t}{2}\left( v_0 - v'_0 \right) - \frac{e^{\alpha t} \sin(\beta t)}{2\beta} \left( v_0 + v'_0 \right);
% \end{align}
% On tire les mêmes conclusions en effectuant un raisonnement similaire.

% \end{proof}





% \subsubsection{Collision inélastique avec séparation des masses}
% \label{subsubsec:colinesepma}

% Reprennons le cas du contact 1D et étudions ce qui se passe durant l'intervale de temps $\tdelta = [\tmoins, \tplus]$ de la collision. Cette fois, pour étudier la dynamique non régulière, nous décidons de séparer les masses $m$ et $m'$ en contact (et ce même durant le contact). Le système résultant est très similaire aux deux cas traités précédemment (\cref{fig:contact1d,fig:contact1d2}), et nous le présentons à la \cref{fig:contact1d3} ci-bas, et son bilan de forces à la \cref{fig:bilan3}.

% \begin{figure}[!h]
%     \centering
%     \frame{\includegraphics[width=0.8\textwidth]{Percussion1D-Systeme-3}}
%     \caption{Contact 1D inélastique entre deux floes. Durant le choc, les nœuds $m$ et $m'$ en contact sont étudiés séparement. On représnte les variables $x_1$, $x_2$, $x_3$, et $x_4$ décrivant les movements de chaque noeud.}
%     \label{fig:contact1d3}
% \end{figure}


% \begin{figure}[!h]
%     \begin{subfigure}[b]{0.30\textwidth}
%         \centering
%         \frame{\includegraphics[width=\textwidth]{Percussion1D3-Masse1.png}}
%         \caption{Sur $m$ de gauche ($x_1$).}
%         \label{fig:bilan13}
%     \end{subfigure}
% %     \hfill
%     \begin{subfigure}[b]{0.35\textwidth}
%         \centering
%         \frame{\includegraphics[width=\textwidth]{Percussion1D3-Masse2}}
%         \caption{Sur $m$ de droite ($x_2$).}
%         \label{fig:bilan23}
%     \end{subfigure}
% %     \hfill
%     \begin{subfigure}[b]{0.35\textwidth}
%         \centering
%         \frame{\includegraphics[width=\textwidth]{Percussion1D3-Masse3}}
%         \caption{Sur $m'$ de gauche ($x_3$).}
%         \label{fig:bilan33}
%     \end{subfigure}
% %     \hfill
%     \begin{subfigure}[b]{0.30\textwidth}
%         \centering
%         \frame{\includegraphics[width=\textwidth]{Percussion1D3-Masse4}}
%         \caption{Sur $m'$ de droite ($x_4$).}
%         \label{fig:bilan43}
%     \end{subfigure}
%        \caption{Bilan des forces appliquée sur les $4$ noeuds du système. On procède de facon similaire aux \cref{fig:bilan,fig:bilan2} pour obtenir les sens et les intensités de ces forces. $F_c$ représente la force de contact dont l'intensité est inconnue.}
%        \label{fig:bilan3}
% \end{figure}

% \noindent Comme précédement, nous appliqons les lois de Newton pour obtenir:
% \begin{align}
%     \begin{dcases}
%     m\ddot x_1 = -k(x_1 - x_2) - \mu (\dot x_1 - \dot x_2) \,, \\
%     m\ddot x_2 = k(x_1 - x_2) + \mu (\dot x_1 - \dot x_2) - F_c \,, \\
%     m'\ddot x_3 = - k'(x_3 - x_4) - \mu'(\dot x_3 - \dot x_4) + F_c \,, \\
%         m' \ddot x_4 =  k'(x_3 - x_4) + \mu'(\dot x_3 - \dot x_4) \,. 
%     \end{dcases}
% \end{align}
% On additionne membre à membre les équations régissant les mouvements de $x_2$ et $x_3$ pour éliminer la force de contact $F_c$ et obtenir le système:
% % \begin{align}\label{eq:systeme3test}
%     \begin{subnumcases}{}
%     m\ddot x_1 = -k(x_1 - x_2) - \mu (\dot x_1 - \dot x_2) \,, \label{eq:sys1D1}\\
%     m\ddot x_2 + m'\ddot x_3 = k(x_1 - x_2) + \mu (\dot x_1 - \dot x_2) - k'(x_3 - x_4) - \mu'(\dot x_3 - \dot x_4) \,, \label{eq:sys1D2} \\
%         m' \ddot x_4 =  k'(x_3 - x_4) + \mu'(\dot x_3 - \dot x_4) \,. \label{eq:sys1D3}
%     \end{subnumcases}
% % \end{align}
% Remarquons que ce système reviens au même systeme étudié dans la partie précédente en posant $x_2(t) = x_3(t) $ p.p. En effet, durant la phase de contact, les massess $m$ et $m'$ peuvent etrs étudiées comme une unique masse $m+m'$. La grosse diffculté qui ressort de cette modélisation est la définitions de la vitesse initiale de l'ensemble $m+m'$. Celà dit, nous cherchons à trouver les vitessses $\dot x_1(\tplus)$, $\dot x_2(\tplus)$, $\dot x_3(\tplus)$ et $\dot x_4(\tplus)$ immédiatemetn après la collision. De par la ressemblance de ce modèle avec celui de la section précédente (voir \cref{eq:systeme1d2}), nous réutilisons les quantités $\dot x_1$ et $\dot x_4$ données par ce système (l'\cref{eq:systeme1d2} dans lequel $x_2$ et $x_3$ sont confondus). On peut se permertre une telle approximation car $x_1$ et $x_4$ n'interviennent pas directemetn dans la collision. De plus, la quantité $k(x_1 - x_2) + \mu (\dot x_1 - \dot x_2) - k'(x_3 - x_4) - \mu'(\dot x_3 - \dot x_4)$
% % \begin{align}
% %     I = k(x_1 - x_2) + \mu (\dot x_1 - \dot x_2) - k'(x_3 - x_4) - \mu'(\dot x_3 - \dot x_4)    
% % \end{align}
% est aussi calculé suivant le modèle \cref{eq:systeme1d2} (voir l'article \parencite{tommasino2020effect} pour une modélisation similaire). Il ne nous reste véritablement que $2$ inconnue dans notre dynamique irrégulière.

% \noindent Intégrons l'équation (\ref{eq:sys1D2}) entre les instants $\tmoins$ et $\tplus$. On obtient:
% \begin{align}    \label{eq:debuteq1}
%     \int_{\tmoins}^{\tplus} m\ddot x_2 + m'\ddot x_3 \diff t = \underbrace{\int_{\tmoins}^{\tplus} k(x_1 - x_2) + \mu (\dot x_1 - \dot x_2) - k'(x_3 - x_4) - \mu'(\dot x_3 - \dot x_4) \diff t}_{I} \,.
% \end{align}
% Afin d'éviter toute confusion, nous notons $v_0 = \dot{x}_2(\tmoins)$ et $v'_0 = \dot{x}_3(\tmoins)$ les vitesses des noeuds en contact avant collision, et $V_0 = \dot{x}_2(\tplus)$ et $V'_0 = \dot{x}_3(\tplus)$ les vitesses après contact. L'équation \cref{eq:debuteq1} devient donc:
% \begin{align} \label{eq:crammer1}
%     mV_0 + m'V'_0 = I + mv_0 + m'v'_0 \,.
% \end{align}
% À présent, nous pouvons étudier l'énergie cinétique du système à travers le coefficient de restitution $\varepsilon$ \footnote{Le coefficient de restitution est le même que celui utilisé dans la thèse \parencite{rabatel2015thesis}.}. On suppose (algébriquement) que les noeuds prennent des directions indiquées à la \cref{fig:contact1dapres}. 
% \begin{figure}[!h]
%     \centering
%     \frame{\includegraphics[width=0.6\textwidth]{Percussion1D3-Apres.png}}
%     \caption{Situation après contact 1D.}
%     \label{fig:contact1dapres}
% \end{figure}

% \noindent On obtient l'\cref{eq:crammer2}:
% \begin{align} \label{eq:crammer2}
%     - V_0 + V'_0 = \varepsilon (v_0 - v'_0) \,.
% \end{align}
% Le système de Cramer qui découle des \cref{eq:crammer1,eq:crammer2} permet d'obtenir les expressions:
% \begin{align} \label{eq:vitessesapres1D}
%     V_0 = \frac{I + (m-\varepsilon m')v_0 + (1+\varepsilon)m'v'_0}{m+m'} \,, \quad V'_0 = \frac{I + (1+\varepsilon)mv_0 + (m'-\varepsilon m)v'_0}{m+m'}\,.
% \end{align}

% % Nous faisons donc ici la grosse hypothèse que \underline{le mouvement de $x_2$ et $x_3$ devient uniforme après la collision}. 

% Une fois leur vitesses "initiales"\footnote{Ces vitesses sont les vitesses de départ pour le deuxième phase de la percussion.} obtenues, on calcule donc les déplacements des différents noeuds des réseaux, et les fractures éventuelles qui s'en suivent. Plus précisément, on a par exmple pour le premier floe:
% \begin{itemize}
%     \item son noeud de gauche $x_1$ a pour vitesse $v_0$ avant et le choc et conserve cette vitesse après le choc;
%     \item son noeud de droite $x_2$ a pour vitesse $v_0$ avant le choc, mais passe de facon discontinue à $V_0$ apres le choc.
% \end{itemize}
% Il en est de même pous le deuxième floe.

% % N'ayant aucune garantie que les vecteurs vitesses $v_0$ et $V_0$ seront opposés immédiatemetn après le choc, nous ne pouvons garantir la convergence de ce modèle (voir \cref{th:div1D}). Ce modèle dégénère (probablement) après la première collision. Effectuons à présent une modélisation 2D et observons si le même problème se repète.

% %%-------Pas vrai
% % Pour $\varepsilon \neq 0$, l'\cref{eq:crammer2} permet de constater que $V_0 = V'_0$ si et seulement si $v_0 = v'_0$. En se basant sur le \cref{th:div1D}, nous pouvons donc énoncer le corrolaire suivant:

% % \begin{corollary} \label{cr:div1D}
% %     Le modèle de collision inélastique 1D avec séparation des masses converge si les vitesses des noeuds avant le choc sont des vecteurs opposés.

% %     Le modèle de collision inélastique 1D avec séparation des masses converge si et seulement si leurs vitesses initiales sont des vecteurs opposés.
% % \end{corollary}
% % \begin{proof}
% %     La preuve découle immédiatement du \cref{th:div1D}.
% % \end{proof}

% % \noindent Le \cref{cr:div1D} permet de constater les limites de notre modélisation 1D. En effet, les moceaux de glace dans la MIZ ne dérivent pas tous à la meme vitesse. Effectuons à présent une modélisation 2D et observons si le même problème se repète.






% \subsubsection{Généralisation et implémentation du modèle 1D}

% Dans cette section, nous généralisons le modèle 1D présenté dans les deux sections précédentes (voir \cref{subsubsec:moddep1D,subsubsec:colinesepma}). Les floes sont cette fois représntés par une multitude de noeuds, de ressorts et de dispositifs visqueux. 

% \paragraph{Déplacement d'un floe.} Considérons la FIGURE CI-BAS où $n$ le nombre de noeuds du floe, $k$ la constante de raideur uniforme de tous ses ressorts, et $\mu$ le coefficient de dissipation pour tous les dispositifs visqueux. 

% FIGURE MODÈLE 1D AVEC PLUSIEURS NOEUDS (NE PAS OUBLIÉ DE REPRÉSENTER LE REPÈRE ABSOLU).

% Contrairement à l'approche par déplacement que nous avons adopté à la \cref{subsubsec:moddep1D}, nous considérons ici une approche par position des noeuds dans le repère absolu (de la figure précédente).  

% RECOPIER LES PAGES 30 ET 31 DU BROUILLON EN RÉSUMANT TANT QUE POSSIBLE



% \paragraph{Etude de la percussion.} Observons que les équations \cref{eq:debuteq1,eq:crammer1,eq:crammer2,eq:vitessesapres1D} de la \cref{subsubsec:colinesepma} restent valident du moment que seuls 2 floes sont en contact. Soit $n_1$ et $n_2$ les nombres de floes respectifs pour les floes de gauche et de droite (voir FIGURE CI-DESSOUS)

% FIGURE DE LA PERCUSSION 1D AVEC PLUSIEURS NOEUDS 


% La figure ci-dessus montre que la quantité $I$ présenté à \cref{eq:debuteq1} s'écrit:

% \noindent Et les vitesses après choc sont données par \cref{eq:vitessesapres1D}\footnote{Notons que les tous les noeuds qui ne sont pas en contact conservent leurs vitesss pendant le choc.}.


% \paragraph{Algorithme de percussion 1D.} Par défnition, la percussion implique de nombreuses collisions entres les floes. Nous présentons donc l'algorithme ci-bas pour déterminer les vitesses et positions des noeuds des floes après la percussion.
% \begin{enumerate}
%     \item Creer deux floes\footnote{Le floe de gauche sera identifié par floe1 et le celui de droite floe2.} tous deux animés de mouvemtn uniforme rectilignes\footnote{Tous les noeuds de chacun des floes ont la même vitesse.}
%     \item Créer le problème de percussion et y ajouter le deux floes et les paramètres phyisiques
%     \item Tant que non collision: - calculer les trajectoire des floes - doubler le temps de simulation avant choc
%     \item Tant que (collision) ou (temps simu atteint) ou (max recussion profondeur): - identifier le moment de collision - calculer les vitesses des noeuds en contact après choc - calculer les trajectoires après choc 
% \end{enumerate}


% \paragraph{diagramme UML du code.} Le code de calcul est conservé dans le FICHIER-REPOSITORY\dots (INDIQUER COMMENT LACER LE CODE). Le diagramme UML définissant les différentes classes et fonctions de notre implémentation est le suivant:

% DIAGRAMME UML DU CODE 1D

% \paragraph{Visualisation des résultats.} 
% PRÉSENTER 5 A 10 TIME STEPS D'UNE SIMULATION AVEC PLUSIEURS FLOES


% \paragraph{Validation du modèle.} Le moyen principal de validation de notre modèle 1D fut l'étude énergétique. AU MOINS LES QUANTITTÉS DE MOUVEMENT ET LES ÉNERGIES TOTALES JUSTE AVANT ET APRÈS CHAQUE CHOCS SONT LES MÊMES.

% PLOT DES QUANTITÉS DE MOUVEMENT ET ENERGIE TOTALES 












%2----------------------------------------------------------------------------------------

\section{Présentation du code de calcul 1D}
DIAGRAMME UML ET README DU REPOSITORY








%3----------------------------------------------------------------------------------------

\section{Résumé des résultats obtenus}

















