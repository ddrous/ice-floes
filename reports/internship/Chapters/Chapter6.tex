% Chapter 6

\chapter{Conclusion} % 6th chapter title

\label{Chapter6} % For referencing the chapter elsewhere, use \ref{Chapter6} 

%----------------------------------------------------------------------------------------


Pour conclure, j'ai effectué mon stage de fin d'étude au Laboratoire Jacques-Louis Lions sous la supervision du professeur Stéphane Labbé. Durant six mois uu sein de ce laboratoire de mathématiques appliqueés, je me suis confonté au problème de la \emph{fracturation des floes de glace} dans un modèle granulaire. J'ai ainsi pu mettre en pratique mes connaissances en calcul scientifique, en analyse numérique, et en développement Python (et bien d'autres) acquises durant ma formation de CSMI. 

A travers les objectifs secondaires que nous nous sommes fixés, j'ai pu comprendre le modèle de rupture de Griffith dans les milieux élastiques en assimilant des domaines clés de l'annalyse fonctionneles tels que la $\Gamma$-convergence, la résonance, et la probabilité); j'ai posé les bases du passage micro/macro du modèle élastique et j'ai developer plusieurs modèles de percussion (1D et 2D). J'ai acquis de l'expérience en développement d’un modèle mathématique et en suivant son intégration à un projet de développement logiciel. Sur un plan technique, j'ai matrisé plusieurs outils inforamtiques tel ques TikZ, Flask, Bokeh, et Symbolab.

Celà dit, il reste plusiueurs travaux à effectuer pour conlclure ce travail. Nous pouvons par exemple citer l'implémentation de la méthode du champ de phase sue le modèle 1D, l'implementation de la fracture au problème 2D, le passage en dimension supérieure (2.5D ou 3D), l'optimisation du code, et des tests de validation en laboratoire.

Fort de cette expérience et ses enjeux cruciaux pour l'indudtrie et pour la planète, j'aimerais apporter ma contribution à la sience à travers ce projet et d'autres similaires. Le projet \href{https://sasip-climate.github.io/}{SASIP} qui a fait de la compréhension du \emph{rechauffement climatique en zone polaire} sa mission phare est une initiative à soutenir et à repéter.