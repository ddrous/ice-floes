% Chapter 6

\chapter{Conclusion} % 6th chapter title

\label{Chapter6} % For referencing the chapter elsewhere, use \ref{Chapter6} 

%----------------------------------------------------------------------------------------


Pour conclure, j'ai effectué mon stage de fin d'étude au Laboratoire Jacques-Louis Lions sous la supervision du professeur Stéphane Labbé. Durant six mois au sein de ce célèbre laboratoire de mathématiques appliquées, je me suis confronté au problème de la \emph{fracturation des floes de glace} dans un modèle granulaire. J'ai ainsi pu mettre en pratique mes connaissances en calcul scientifique, en analyse numérique, en EDP, en analyse fonctionnelle, et en développement Python (et bien d'autres) acquises durant ma formation de \href{https://www.unistra.fr/etudes/decouvrir-nos-formations/par-facultes-ecoles-instituts/sciences-technologies/ufr-de-mathematique-et-dinformatique/ufr-de-mathematique-et-dinformatique/cursus/ME195?cHash=3aa4f04702a03e944ed933056abe17f2}{CSMI}. 

À travers les objectifs secondaires que nous nous sommes fixés, j'ai pu comprendre le modèle de rupture de Griffith dans les milieux élastiques en assimilant des domaines clés de l'analyse fonctionnelle tels que la $\Gamma$-convergence, la résonance, et la probabilité; j'ai posé les bases du passage micro/macro du modèle élastique et j'ai développé plusieurs modèles de percussion (1D et 2D). J'ai acquis de l'expérience en programmation d’un modèle mathématique par différents schémas numériques, et en suivi de son intégration à un projet de développement logiciel. Sur un plan technique, j'ai maitrisé plusieurs outils informatiques tels que TikZ, Flask, Bokeh, et Symbolab.

Cela dit, il reste plusieurs travaux à effectuer pour conclure ce travail. Nous pouvons par exemple citer l'implémentation de la méthode du champ de phase dans le modèle 1D, l'implémentation de la fracture dans le problème 2D, le passage en dimension supérieure (2.5D ou 3D), l'optimisation du code, des tests de confirmation de l'approximation par réseaux de ressorts à grande raideur, et des tests de validation des modèles en laboratoire.

Fort de cette expérience et de ses enjeux cruciaux pour l'industrie et pour la planète, j'aimerais apporter ma contribution à la science à travers ce projet et d'autres similaires. Le projet \href{https://sasip-climate.github.io/}{SASIP} qui a fait de la compréhension du \emph{réchauffement climatique en zone polaire} sa mission phare est une initiative à soutenir et à répéter.