% Chapter 4

\chapter{Travaux et apports} % 4th chapter title

\label{Chapter4} % For referencing the chapter elsewhere, use \ref{Chapter4} 


%----------------------------------------------------------------------------------------


\section{Les missions du poste}

\begin{itemize}
    \item L'état de l'art de la partie précédente fait partie des missions.
\end{itemize}

Nous souhaitons étudier le comportement mécanique d'un floe après collision avec un autre floe. Les étapes de travail envisagées sont les suivantes:
\begin{enumerate}
    \item Ecire les systèmes differentiels pour les deux floes juste après le choc: ca donne 4 EDO dont 2 SE, et 2 SI. Pour l' instant on peut considérer que l'un des floes est immobile (celà revient au même si l'on exprimes les vitesses dans un repère lié à ce floe).
    \item On exprime l'EDO vérifiée par les solutions, c'est à dire $q$ pour le premier floes, et $p$ pour le second.
    \item On pourra ensuite simuler ces EDP limites et trouver les valeurs de $p$ et $q$. Autrement dit, on connait la position de chaque point du réseau au temps final.
    \item Si on connait $p$ et/ou $q$, on connait la condition de Dirichlet sur le floe concerné, et on peut ainsi exprimer le déplacement et la possible fracture du matériau. 
\end{enumerate}




\section{Présentation des résulats obtenus}




\section{Les apports du stage}

\begin{itemize}
    \item L' utilisation de TIKZ
\end{itemize}