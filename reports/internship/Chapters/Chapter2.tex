% Chapter 2

\chapter{Contexte, environnement et objectifs} % 2 nd chapter title

\label{Chapter2} % For referencing the chapter elsewhere, use \ref{Chapter2} 


%1----------------------------------------------------------------------------------------

\section{Contexte}

PRÉSENTER LE LABORATOIRE LJLL AVEC QUELQUES DÉTAILS











%2----------------------------------------------------------------------------------------

\section{Roadmap}











%3----------------------------------------------------------------------------------------

\section{Objectifs et missions du poste}






% \subsection{Objectifs}
% INDIQUER ICI QUE LES TRAVAUX SERONT PRÉSENTÉS EN DETAILS DANS LES SECTIONS 2, 3, ET 4.









% \subsection{Les missions du poste}


% %%------------------- Cette partie doit etre intégrée à la suivante
% Au vue du problème qu'il nous est donné de résoudre et des travaux qui ont précédés, les taches suivantes (classées par ordre de priorité) ont été effectuées durant ce stage :

% \begin{enumerate}
%     \item Lecture des travaux de Dimitri et Matthias (des petits résumés)
%     \item Modéliser, simuler, etc..
%     \item Débugger et améliorer l'interface web de Dimitri:
%     \begin{itemize}
%         \item Correction du "collapsing" du Displacement Field
%         \item Correction du slider pour changer les images d'eigen vecteurs
%     \end{itemize}
% \end{enumerate}
% %%-------------------------- Citer ces taches sous forme de paragraphe LateX




% \begin{itemize}
%     \item L'état de l'art de la partie précédente fait partie des missions.
%     \item Modélisation
%     \item Simulation
% \end{itemize}

% Nous souhaitons étudier le comportement mécanique d'un floe après collision avec un autre floe. Les étapes de travail envisagées sont les suivantes:
% \begin{enumerate}
%     \item Ecire les systèmes differentiels pour les deux floes juste après le choc: pour l' instant on peut considérer que l'un des floes est immobile (celà revient au même si l'on exprimes les vitesses dans un repère lié à ce floe).
%     \item On exprime l'EDO vérifiée par les solutions, c'est à dire $q$ pour le premier floes, et $p$ pour le second.
%     \item On pourra ensuite simuler ces EDP limites et trouver les valeurs de $p$ et $q$. Autrement dit, on connait la position de chaque point du réseau au temps final.
%     \item Si on connait $p$ et/ou $q$, on connait la condition de Dirichlet sur le floe concerné, et on peut ainsi exprimer le déplacement et la possible fracture du floe. 
% \end{enumerate}


