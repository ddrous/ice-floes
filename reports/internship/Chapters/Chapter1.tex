% Chapter 1

\chapter{Introduction} % Main chapter title

\label{Chapter1} % For referencing the chapter elsewhere, use \ref{Chapter1} 

%----------------------------------------------------------------------------------------

% Define some commands to keep the formatting separated from the content 
\newcommand{\keyword}[1]{\textbf{#1}}
\newcommand{\tabhead}[1]{\textbf{#1}}
\newcommand{\code}[1]{\texttt{#1}}
\newcommand{\file}[1]{\texttt{\bfseries#1}}
\newcommand{\option}[1]{\texttt{\itshape#1}}






%1----------------------------------------------------------------------------------------



\section{Contexte}



Le déclin de la glace Artique ces dernières décenies est considéré comme l'une manifestations les plus marquante du changement climatique (voir \parencite{stroeve2012trends}). Ce déclin présente des enjeux aussi climatiques qu'industriels. Premièrement, de par son étendue et son epaisseur immense, la la zone artique est un contributeur majeur climat à travers ses échanges de chaleurs par rayonnement et radiation avec l'atmosphère. Il est donc crucial de considérer l'évolution de la glace dans les modèles climatiques. Deuxièmement, la chute de cette couverture de glace dans la MIZ\footnote{Marginal Ice Zone : zone de transition entre l’océan et le coeur de la banquise, où la concentration de
glace est inférieure à 80\%, et/ou les morceaux de glace sont de faible épaisseur ($\approx 1 mètre$) et de petite taille ($10 m - 100 km$).} (VOIR FIGURE) ouvre des routes maritimes facilitant l'exploitation de ses réserves d’hydrocarbures (qui restent quasiment intactes). Il est donc nécéssaire de pouvoir prédire l'évolution de la banquise Artique (au moins) à court terme.


Parmis les élément exhaxerbant ce déclin de galce Artique, des études ont cité l'accélération de la vitesse et de la déformation des floes\footnote{Un floe est un morceau indiviiduel de glace rencontré dans la MIZ} (RWDC11, SKM11). Pour les prédiction de l'évolution de la banquise, les modèles qui considère la glace comme un milieu continu ne sont pas adapté, surtout à l'échelle de la MIZ. Au contraire, les modèle granulaire, bien que plus couteux doivent etre priviligé afin de prendre en compte la nature discontinue de la banquise et sa rhéologie\footnote{étudie la résistance des matériaux aux contraintes et aux déformations.}. Des modèles granulaires pour l'évolution de la glace par le passé (Hop96, KS14). Cependant, les approches utulisées dans ces travaux limitent la géométrie (circulaire, rectangulaire) et le nombre de floes (de l'ordre de la centaine) (voir \parencite[p.16]{balasoiu2020halthesis}).


SASIP 



%2----------------------------------------------------------------------------------------





\section{Problématique et missions}








%3----------------------------------------------------------------------------------------





\section{Environnement}

LJLL, GRENOBLE et Teletravail





%4----------------------------------------------------------------------------------------





\section{Roadmap}
Missions (objectifs primaire) et secondaires. Ensuite les milestones. Enfin annonce du plan (qui résument les travaux)




% \subsection{Objectifs}
% INDIQUER ICI QUE LES TRAVAUX SERONT PRÉSENTÉS EN DETAILS DANS LES SECTIONS 2, 3, ET 4.









% \subsection{Les missions du poste}


% %%------------------- Cette partie doit etre intégrée à la suivante
% Au vue du problème qu'il nous est donné de résoudre et des travaux qui ont précédés, les taches suivantes (classées par ordre de priorité) ont été effectuées durant ce stage :

% \begin{enumerate}
%     \item Lecture des travaux de Dimitri et Matthias (des petits résumés)
%     \item Modéliser, simuler, etc..
%     \item Débugger et améliorer l'interface web de Dimitri:
%     \begin{itemize}
%         \item Correction du "collapsing" du Displacement Field
%         \item Correction du slider pour changer les images d'eigen vecteurs
%     \end{itemize}
% \end{enumerate}
% %%-------------------------- Citer ces taches sous forme de paragraphe LateX




% \begin{itemize}
%     \item L'état de l'art de la partie précédente fait partie des missions.
%     \item Modélisation
%     \item Simulation
% \end{itemize}

% Nous souhaitons étudier le comportement mécanique d'un floe après collision avec un autre floe. Les étapes de travail envisagées sont les suivantes:
% \begin{enumerate}
%     \item Ecire les systèmes differentiels pour les deux floes juste après le choc: pour l' instant on peut considérer que l'un des floes est immobile (celà revient au même si l'on exprimes les vitesses dans un repère lié à ce floe).
%     \item On exprime l'EDO vérifiée par les solutions, c'est à dire $q$ pour le premier floes, et $p$ pour le second.
%     \item On pourra ensuite simuler ces EDP limites et trouver les valeurs de $p$ et $q$. Autrement dit, on connait la position de chaque point du réseau au temps final.
%     \item Si on connait $p$ et/ou $q$, on connait la condition de Dirichlet sur le floe concerné, et on peut ainsi exprimer le déplacement et la possible fracture du floe. 
% \end{enumerate}






%5----------------------------------------------------------------------------------------




\section{Résumé de l'introduction en Anglais}