%%%%%%%%%%%%%%%%%%%%%%%%%%%%%%%%%%%%%%%%%
% fphw Assignment
% LaTeX Template
% Version 1.0 (27/04/2019)
%
% This template originates from:
% https://www.LaTeXTemplates.com
%
% Authors:
% Class by Felipe Portales-Oliva (f.portales.oliva@gmail.com) with template 
% content and modifications by Vel (vel@LaTeXTemplates.com)
%
% Template (this file) License:
% CC BY-NC-SA 3.0 (http://creativecommons.org/licenses/by-nc-sa/3.0/)
%
%%%%%%%%%%%%%%%%%%%%%%%%%%%%%%%%%%%%%%%%%

%----------------------------------------------------------------------------------------
%	PACKAGES AND OTHER DOCUMENT CONFIGURATIONS
%----------------------------------------------------------------------------------------

\documentclass[
  french,
  % twocolumn,
	11pt, % Default font size, values between 10pt-12pt are allowed
	%letterpaper, % Uncomment for US letter paper size
	%spanish, % Uncomment for Spanish
]{fphw}

% \usepackage[fontsize=10.0]{scrextend} % Use this to force the fontsize

%% Commands for numering paragraphs
\renewcommand\thesection{\Roman{section}}
\renewcommand\thesubsection{\thesection.\arabic{subsection}}
\renewcommand*\thesubsubsection{%
  \Roman{section}.\arabic{subsection}.\alph{subsubsection}%
}

 % \usepackage{fancyhdr}
% Template-specific packages
\usepackage{babel}
\usepackage[utf8]{inputenc} % Required for inputting international characters
% \usepackage{DejaVuSerifCondensed} 
\usepackage[T1]{fontenc} % Output font encoding for international characters
% \usepackage{mathpazo} % Use the Palatino font
% \usepackage{tgschola} % Use the Palatino font
% \usepackage{Alegreya}
% \renewcommand*\oldstylenums[1]{{\AlegreyaOsF #1}}
% \usepackage{iwona} % Use the Iwona font

\usepackage{kpfonts}        %% For math only
\usepackage{fontspec}       %% Because we are using XeTEX
\setromanfont{Minion Pro}   %% For text (Minion Math is commercial)

%-----------------------------------------------------------------------
% \setromanfont{Meta Serif Pro}
% \setsansfont{Fira Sans}
% \setmonofont[Color={0019D4}]{Fira Code} 
%-----------------------------------------------------------------------



\usepackage{fancyvrb}
\usepackage{fvextra}
\newcommand\userinput[1]{\textbf{#1}}
\newcommand\arguments[1]{\textit{#1}}

\usepackage{amsmath}
\usepackage{mathtools}
\usepackage{xfrac} 

\usepackage{graphicx} % Required for including images
\usepackage[textfont=it,font=small]{caption}  %% To manage long captions in images
\usepackage{subcaption}
\captionsetup{justification=centering}

\usepackage{float}
\graphicspath{ {./img/} }

\usepackage{booktabs} % Required for better horizontal rules in tables

\usepackage{listings} % Required for insertion of code

\usepackage{array} % Required for spacing in tabular environment

\usepackage{enumerate} % To modify the enumerate environment

\usepackage{amssymb}
\usepackage{enumitem}	%% % To modify the itemize bullet character

\newcommand{\tabhead}[1]{{\bfseries#1}}

\usepackage{xcolor}
\usepackage{listings}
\colorlet{mygray}{black!30}
\colorlet{mygreen}{green!60!blue}
\colorlet{mymauve}{red!60!blue}
\lstset{
  backgroundcolor=\color{gray!10},  
  basicstyle=\ttfamily,
  columns=fullflexible,
  breakatwhitespace=false,      
  breaklines=true,                
  captionpos=b,                    
  commentstyle=\color{mygreen}, 
  extendedchars=true,              
  frame=single,                   
  keepspaces=true,             
  keywordstyle=\color{blue},      
  language=c++,                 
  numbers=none,                
  numbersep=5pt,                   
  numberstyle=\tiny\color{blue}, 
  rulecolor=\color{mygray},        
  showspaces=false,               
  showtabs=false,                 
  stepnumber=5,                  
  stringstyle=\color{mymauve},    
  tabsize=3,                      
  title=\lstname                
}

\usepackage[linkcolor=blue,colorlinks=true]{hyperref}
% \usepackage[colorlinks=true,urlcolor=blue]{hyperref}
\hypersetup{citecolor=blue}

\usepackage{cleveref}
\usepackage{siunitx}
\newcommand{\bvec}[1]{\bm{#1}}    %% For vector notation
\newcommand{\myvec}[3]{\begin{pmatrix} #1  \\ #2 \\ #3 \end{pmatrix}}   %% vecteur 3d
\newcommand{\mymat}[9]{\begin{pmatrix} #1 & #2 & #3 \\ #4 & #5 & #6 \\ #7 & #8 &#9 \end{pmatrix}}  %% Matrice 3*3

\renewcommand{\vector}[4]{\begin{pmatrix} #1  \\ #2 \\ #3 \\ #4 \end{pmatrix}}   %% vecteur 3d
% \newcommand{\mymatrix}[16]{\begin{pmatrix} #1 & #2 & #3 & #4 \\ #4 & #6 & #7 & #8 \\ #9 & #10 & #11 & #12 \\ #13 & #14 & #15 & #16 \end{pmatrix}}  %% Matrice 3*3

\newcommand{\hquad}{\hspace{0.5em}} %% Bew command for half quad
% \setlength\parindent{0pt}	%% To remove all indentations
% \setlength{\parskip}{1em}%

\usepackage[backend=bibtex,style=authoryear,maxnames=2,natbib=true]{biblatex} % Use the bibtex backend with the authoryear citation style (which resembles APA)
\addbibresource{bibliography.bib} % The filename of the bibliography
\usepackage[autostyle=true]{csquotes} % Required to generate language-dependent quotes in the bibliography 
% \renewcommand*{\bibfont}{\tiny} % Pour reduire la taille des references

%----------------------------------------------------------------------------------------
%	ASSIGNMENT INFORMATION
%----------------------------------------------------------------------------------------

\title{Résumé thèse M. Rabatel} % Assignment title

\author{Desmond Roussel Nzoyem} % Student name

\date{\today} % Due date

\institute{Université de Strasbourg, Sorbone Université \\ Laboratoire Jacques-Louis Lions} % Institute or school name

\class{Stage M2} % Course or class name

\professor{Pr. Stéphane Labbé} % Professor or teacher in charge of the assignment

%----------------------------------------------------------------------------------------

\begin{document}

\maketitle % Output the assignment title, created automatically using the information in the custom commands above

%----------------------------------------------------------------------------------------
%	ASSIGNMENT CONTENT - CODE COULEUR
%----------------------------------------------------------------------------------------
\section*{Code couleur}
\begin{description}
  \item[\textcolor{yellow}{jaune}]: information importante;
  \item[\textcolor{blue}{blue}]: information triviale;
  \item[\textcolor{green}{vert}]: question à poser;
  \item[\textcolor{red}{rouge}]: possible ereur;
  \item[\textcolor{brown}{marron}]: piste d'étude pour le stage;
\end{description}

%----------------------------------------------------------------------------------------
%	ASSIGNMENT CONTENT - INTRODUCTION
%----------------------------------------------------------------------------------------

\section*{Introdution}

La simulation de la glace artique est importante pour:
\begin{itemize}
  \item les prédictions climatiques, vu le role que joue la coucle dans la reflection des rayonnements.
  \item l'exploitation des ressources, suite à la fonte de la glace artique
\end{itemize}

Le cout numérique élevé des modèles granulaires justifie leur rareté. Mais il y a un regain ces dernière années. 

L'algorithme ne repose pas sur les méthodes classiques de dynamique moléculaire\footnote{Technique de simulation numérique permettant de modéliser l'évolution d'un système de particules au cours du temps.}; mais à partir d’un algorithme de gestion d’événements\footnote{La gestion d’événements (ou Event-Driven) consiste à séparer la dynamique régulière de la dynamique non régulière, comme les collisions.} et avec une attention particulière sur les collisions entre les floes, tout en évitant les interpénétrations\footnote{lorsque la distance entre deux floes est négative i.e. $\delta \leq 0$. Il faut g gérer les interpénétrations avant qu'elles n'arrivent.}. 


%----------------------------------------------------------------------------------------
%	ASSIGNMENT CONTENT - SECTION 1
%----------------------------------------------------------------------------------------


\section{Modélisation Théorique de la Dynamique des Glaces de Mer}

\subsection{Le modèle du floe}


\begin{itemize}
  \item Les floes de glace sont à peine connexes. 
  \item Le modèle considère des floes dont l’aire est supérieure à quelques mètres carrés et sans restrictions concernant leur géométrie. À cette échelle, l’épaisseur et les processus hors-plan peuvent-être négligés en première approximation.
  \item L'échelle temporelle est celle de quelques jours;
  \item Les obstacles sont des floes de masse volumique très forte;
  \item Le coefficient de restitution $\varepsilon$ détermine le caratère élastique ou inélastique de la collision;
  \item Le repère absolu $\mathcal{R}_{abs}$ est euclidien, ce qui empèche de prendre en compte la rotondité de la terre (rondeur);
  \item L'aipaisseur $h$ du floe peut varier en espace, mais pas en temps, vu que les floes sont considéres rigide, et on a négliger les effets thermodynamiques.
  \item La forme d’une condition de complémentarité en vitesse impulsion pour garantir l’existence de solutions au problème du contact.
  \item Les équations du moment linéaire et du moment angulaire en vitesse-impulsion; car il y a l’avantage de pouvoir exprimer la force de friction de Coulomb directement par rapport à la vitesse. Il n’est pas nécessaire de connaître la nature
  du contact.
  \item Les lois de contacts utilisées sont des lois implicites \footnote{Une loi de contact est explicites quand les réactions sont connues et calculées en fonction de données comme la distance entre les floes et implicites quand les forces ou les réactions sont inconnues.} afin de chercher une formulation en vitesse impulsion et conserver une physique réaliste du contact (Pas de loi de Hertz, ni Coulomb, etc.) (page 31)
  \item La loi de conntact utilisée loi est appelée condition unilatérale car elle permet aux floes de se séparer, mais interdit l’interpénétration (p. 32).
  \item Grâce à la formulation en problème linéaire de complémentarité, il a montré capables de montrer qu’il existe des vitesses solutions pour toutes configurations de contact avec friction de Coulomb (p. 38). 
\end{itemize}

%----------------------------------------------------------------------------------------
%	ASSIGNMENT CONTENT - CONCLUSION
%----------------------------------------------------------------------------------------

\section*{Conclusion}

Les points forts du modèle:
\begin{enumerate}
  \item Ce modèle considère des tailles de glace réalistes, aussi petites ($\leq 10 km$) que grandes ($\geq 10 km$); des concentrations variées (plus ou moins de $80 \%$ de glace), etc.
  \item Ce modèle gère l'interpénétration des floes: on ne veut pas qu'un morceau de glace ratre dans un autre;
\end{enumerate}


Les points faibles:
\begin{enumerate}
  \item ne gère pas la rhéologie\footnote{La rhéologie est l'étude de la déformation et de l'écoulement de la matière sous l'effet d'une contrainte appliquée.} de la glace: les floes sont des solides purement rigides (ils ne se déforment pas) et la dissipation d’énergie cinétique durant la collision est décrite en utilisant un coefficient purement empirique
\end{enumerate}



% %-------------------------------------------------------------------------------
% %							THE BIBLIOGRAPHY
% %-------------------------------------------------------------------------------
% \clearpage   % Pour retirer les references de la bare de navigation
% % \cite{CNRS} \cite{Kaggle} \cite{GitHub} \cite{JHU} \cite{Nguemdjo}
% \printbibliography



% %-------------------------------------------------------------------------------
% %							THE BIBLIOGRAPHY
% %-------------------------------------------------------------------------------
% \clearpage   % Pour retirer les references de la bare de navigation
% % \cite{CNRS} \cite{Kaggle} \cite{GitHub} \cite{JHU} \cite{Nguemdjo}
% \printbibliography


\end{document}
