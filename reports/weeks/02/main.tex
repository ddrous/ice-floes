%%%%%%%%%%%%%%%%%%%%%%%%%%%%%%%%%%%%%%%%%
% fphw Assignment
% LaTeX Template
% Version 1.0 (27/04/2019)
%
% This template originates from:
% https://www.LaTeXTemplates.com
%
% Authors:
% Class by Felipe Portales-Oliva (f.portales.oliva@gmail.com) with template 
% content and modifications by Vel (vel@LaTeXTemplates.com)
%
% Template (this file) License:
% CC BY-NC-SA 3.0 (http://creativecommons.org/licenses/by-nc-sa/3.0/)
%
%%%%%%%%%%%%%%%%%%%%%%%%%%%%%%%%%%%%%%%%%

%----------------------------------------------------------------------------------------
%	PACKAGES AND OTHER DOCUMENT CONFIGURATIONS
%----------------------------------------------------------------------------------------

\documentclass[
  french,
  % twocolumn,
	11pt, % Default font size, values between 10pt-12pt are allowed
	%letterpaper, % Uncomment for US letter paper size
	%spanish, % Uncomment for Spanish
]{fphw}

% \usepackage[fontsize=10.0]{scrextend} % Use this to force the fontsize

%% Commands for numbering paragraphs
\renewcommand\thesection{\Roman{section}}
\renewcommand\thesubsection{\thesection.\arabic{subsection}}
\renewcommand*\thesubsubsection{%
  \Roman{section}.\arabic{subsection}.\alph{subsubsection}%
}

% Template-specific packages
\usepackage{babel}
\usepackage[utf8]{inputenc} % Required for inputting international characters
% \usepackage{DejaVuSerifCondensed} 
\usepackage[T1]{fontenc} % Output font encoding for international characters

\usepackage{kpfonts}        %% For math only
\usepackage{fontspec}       %% Because we are using XeTEX
\setromanfont{Minion Pro}   %% For text (Minion Math is commercial)

%-----------------------------------------------------------------------
% \setromanfont{Meta Serif Pro}
% \setsansfont{Fira Sans}
% \setmonofont[Color={0019D4}]{Fira Code} 
%-----------------------------------------------------------------------

\usepackage{fancyvrb}
\usepackage{fvextra}
\newcommand\userinput[1]{\textbf{#1}}
\newcommand\arguments[1]{\textit{#1}}

\usepackage{amsmath}
\usepackage{mathtools}
\usepackage{xfrac} 

\usepackage{graphicx} % Required for including images
\usepackage[textfont=it,font=small]{caption}  %% To manage long captions in images
\usepackage{subcaption}
\captionsetup{justification=centering}

\usepackage{float}
\graphicspath{ {../img/} }

\usepackage{booktabs} % Required for better horizontal rules in tables

\usepackage{listings} % Required for insertion of code

\usepackage{array} % Required for spacing in tabular environment

\usepackage{enumerate} % To modify the enumerate environment

\usepackage{amssymb}
\usepackage{enumitem}	%% % To modify the itemize bullet character

\newcommand{\tabhead}[1]{{\bfseries#1}}

\usepackage{xcolor}
\usepackage{listings}
\colorlet{mygray}{black!30}
\colorlet{mygreen}{green!60!blue}
\colorlet{mymauve}{red!60!blue}
\lstset{
  backgroundcolor=\color{gray!10},  
  basicstyle=\ttfamily,
  columns=fullflexible,
  breakatwhitespace=false,      
  breaklines=true,                
  captionpos=b,                    
  commentstyle=\color{mygreen}, 
  extendedchars=true,              
  frame=single,                   
  keepspaces=true,             
  keywordstyle=\color{blue},      
  language=c++,                 
  numbers=none,                
  numbersep=5pt,                   
  numberstyle=\tiny\color{blue}, 
  rulecolor=\color{mygray},        
  showspaces=false,               
  showtabs=false,                 
  stepnumber=5,                  
  stringstyle=\color{mymauve},    
  tabsize=3,                      
  title=\lstname                
}

\usepackage[linkcolor=blue,colorlinks=true]{hyperref}
% \usepackage[colorlinks=true,urlcolor=blue]{hyperref}
\hypersetup{citecolor=blue}

\usepackage{cleveref}
\usepackage{siunitx}
\usepackage{bm}

\usepackage[backend=bibtex,style=alphabetic,maxnames=2,natbib=true]{biblatex} % Use the bibtex backend with the alphabetic citation style (compact APA-like)
% \usepackage[backend=bibtex,style=authoryear,maxnames=2,natbib=true]{biblatex} % Use the bibtex backend with the authoryear citation style (which resembles APA)
\addbibresource{../bib/bibliography.bib} % The filename of the bibliography
\usepackage[autostyle=true]{csquotes} % Required to generate language-dependent quotes in the bibliography 
% \renewcommand*{\bibfont}{\tiny} % Pour reduire la taille des references

\usepackage[useregional=numeric]{datetime2}
\usepackage[normalem]{ulem}

% %-------------------------------------------------------------------------------

\newcommand{\myvec}[3]{\begin{pmatrix} #1  \\ #2 \\ #3 \end{pmatrix}}   %% vecteur 3d
\newcommand{\mymat}[9]{\begin{pmatrix} #1 & #2 & #3 \\ #4 & #5 & #6 \\ #7 & #8 &#9 \end{pmatrix}}  %% Matrice 3*3

\renewcommand{\vector}[4]{\begin{pmatrix} #1  \\ #2 \\ #3 \\ #4 \end{pmatrix}}   %% vecteur 4d
% \newcommand{\mymatrix}[16]{\begin{pmatrix} #1 & #2 & #3 & #4 \\ #4 & #6 & #7 & #8 \\ #9 & #10 & #11 & #12 \\ #13 & #14 & #15 & #16 \end{pmatrix}}  %% Matrice 3*3

\newcommand{\hquad}{\hspace{0.5em}} %% Bew command for half quad
\newcommand*\diff{\mathop{}\!\mathrm{d}}
% \setlength\parindent{0pt}	%% To remove all indentations
\newcommand{\bvec}[1]{\bm{\mathrm{#1}}}  %% Use this to make vectors
\newcommand{\bmat}[1]{\bm{\mathsf{#1}}}   %% Use this to make tensors

%----------------------------------------------------------------------------------------
%	ASSIGNMENT INFORMATION
%----------------------------------------------------------------------------------------

\title{Compte rendu semaine \#2} % Assignment title
% \title{Difficultés rencontrées} % Assignment title

\author{Desmond Roussel Nzoyem} % Student name

\date{\DTMdisplaydate{2021}{2}{10}{-1} - \DTMdisplaydate{2021}{2}{16}{-1}} % Due date

\institute{Sorbonne Université \\ Laboratoire Jacques-Louis Lions} % Institute or school name

\class{Stage M2} % Course or class name

\professor{Pr. Stéphane Labbé} % Professor or teacher in charge of the assignment

%----------------------------------------------------------------------------------------

\begin{document}

\maketitle % Output the assignment title, created automatically using the information in the custom commands above

%----------------------------------------------------------------------------------------
%	ASSIGNMENT CONTENT - INTRO
%----------------------------------------------------------------------------------------


L'objectif de cette semaine fut la compréhension complète du modèle de dynamique des floes de glace rigides développé dans la thèse de M. Rabatel \parencite{rabatel2015thesis}. L'objectif a globalement été atteint, même si quelques points restent non résolus. Certains de ces points (par exemple la définition des conditions aux bords, les interactions dans la ZIR, etc.) ne pourront être clairement compris qu'après analyse du code de calcul.


%----------------------------------------------------------------------------------------
%	ASSIGNMENT CONTENT - SECTION 1
%----------------------------------------------------------------------------------------

\section{Tâches effectuées}

\begin{enumerate}
  \item Lecture du chapitre 2 de la thèse de M. Rabatel \parencite{rabatel2015thesis}.
  \item Lecture du chapitre 3 de la thèse de M. Rabatel \parencite{rabatel2015thesis}.
  \item Début de lecture de la thèse de D. Balasoiu \parencite{balasoiu2020thesis}.
  \item Continuation de rédaction de l'état de l'art pour le rapport de stage.
  \item Rédaction du rapport pour la semaine $2$ de stage.
\end{enumerate}


%----------------------------------------------------------------------------------------
%	ASSIGNMENT CONTENT - SECTION 2
%----------------------------------------------------------------------------------------

\section{Difficultés rencontrées}

\textbf{\textit{Les questions suivantes sont totalement ou partiellement inspirées de la thèse de M. Rabatel. Celles qui sont barrées sont celles pour lesquelles une réponse a été obtenue.}}

\begin{enumerate}
  \item \sout{Le problème de complémentarité avec la loi de Newton : équation (1.1.27) \parencite[p.41]{rabatel2015thesis}. Je ne comprends plus sa signification lorsqu'on ajoute le terme $\varepsilon J^T W(t^{-})$.}
  \item À la figure 1.6 \parencite[p.43]{rabatel2015thesis}, est-il possible de calculer l'énergie cinétique $E(t^+)$ sans avoir au préalable calculé $W(t^+)$ ?
  \item Dans les cas $(iv)$ de \parencite[p.79]{rabatel2015thesis}, le contact est ponctuel et $\bvec{PQ} \notin \mathcal{N}_P$. Mais ce cas semble ne pas être considéré dans les simulations d'après le dernier paragraphe de la page $80$. Es-ce bien le cas ?
\end{enumerate}



%----------------------------------------------------------------------------------------
%	ASSIGNMENT CONTENT - SECTION 3
% ----------------------------------------------------------------------------------------

\section{Sujets explorables}

\begin{enumerate}
  \item Rajouter le caractère déformable aux floes dans le modèle granulaire \parencite[p.12]{rabatel2015thesis}.
  \item Vu que le coefficient de restitution est fixé de façon empirique, on pourrait déterminer ce coefficient en fonction de l’endommagement subi par le floe et des éventuelles fractures se propageant dans le floe après une situation de collision \parencite[p.14]{rabatel2015thesis}.
  \item Intégrer la notion de contact actif \parencite[p.86]{rabatel2015thesis} dans le modèle, et refaire le test du berceau de Newton \parencite[p.108]{rabatel2015thesis}.
\end{enumerate}

% %-------------------------------------------------------------------------------
% %							THE BIBLIOGRAPHY
% %-------------------------------------------------------------------------------
\clearpage   % Pour retirer les references de la bare de navigation
\printbibliography


\end{document}
