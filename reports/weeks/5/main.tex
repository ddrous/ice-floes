%%%%%%%%%%%%%%%%%%%%%%%%%%%%%%%%%%%%%%%%%
% fphw Assignment
% LaTeX Template
% Version 1.0 (27/04/2019)
%
% This template originates from:
% https://www.LaTeXTemplates.com
%
% Authors:
% Class by Felipe Portales-Oliva (f.portales.oliva@gmail.com) with template 
% content and modifications by Vel (vel@LaTeXTemplates.com)
%
% Template (this file) License:
% CC BY-NC-SA 3.0 (http://creativecommons.org/licenses/by-nc-sa/3.0/)
%
%%%%%%%%%%%%%%%%%%%%%%%%%%%%%%%%%%%%%%%%%

%----------------------------------------------------------------------------------------
%	PACKAGES AND OTHER DOCUMENT CONFIGURATIONS
%----------------------------------------------------------------------------------------

\documentclass[
  french,
  % twocolumn,
	11pt, % Default font size, values between 10pt-12pt are allowed
	%letterpaper, % Uncomment for US letter paper size
	%spanish, % Uncomment for Spanish
]{fphw}

% \usepackage[fontsize=10.0]{scrextend} % Use this to force the fontsize

%% Commands for numbering paragraphs
\renewcommand\thesection{\Roman{section}}
\renewcommand\thesubsection{\thesection.\arabic{subsection}}
\renewcommand*\thesubsubsection{%
  \Roman{section}.\arabic{subsection}.\alph{subsubsection}%
}

% Template-specific packages
\usepackage{babel}
\usepackage[utf8]{inputenc} % Required for inputting international characters
% \usepackage{DejaVuSerifCondensed} 
\usepackage[T1]{fontenc} % Output font encoding for international characters

\usepackage{kpfonts}        %% For math only
\usepackage{fontspec}       %% Because we are using XeTEX
\setromanfont{Minion Pro}   %% For text (Minion Math is commercial)

%-----------------------------------------------------------------------
% \setromanfont{Meta Serif Pro}
% \setsansfont{Fira Sans}
% \setmonofont[Color={0019D4}]{Fira Code} 
%-----------------------------------------------------------------------

\usepackage{fancyvrb}
\usepackage{fvextra}
\newcommand\userinput[1]{\textbf{#1}}
\newcommand\arguments[1]{\textit{#1}}

\usepackage{amsmath}
\usepackage{mathtools}
\usepackage{xfrac} 

\usepackage{graphicx} % Required for including images
\usepackage[textfont=it,font=small]{caption}  %% To manage long captions in images
\usepackage{subcaption}
\captionsetup{justification=centering}

\usepackage{float}
\graphicspath{ {../img/} }

\usepackage{booktabs} % Required for better horizontal rules in tables

\usepackage{listings} % Required for insertion of code

\usepackage{array} % Required for spacing in tabular environment

\usepackage{enumerate} % To modify the enumerate environment

\usepackage{amssymb}
\usepackage{enumitem}	%% % To modify the itemize bullet character

\newcommand{\tabhead}[1]{{\bfseries#1}}

\usepackage{xcolor}
\usepackage{listings}
\colorlet{mygray}{black!30}
\colorlet{mygreen}{green!60!blue}
\colorlet{mymauve}{red!60!blue}
\lstset{
  backgroundcolor=\color{gray!10},  
  basicstyle=\ttfamily,
  columns=fullflexible,
  breakatwhitespace=false,      
  breaklines=true,                
  captionpos=b,                    
  commentstyle=\color{mygreen}, 
  extendedchars=true,              
  frame=single,                   
  keepspaces=true,             
  keywordstyle=\color{blue},      
  language=c++,                 
  numbers=none,                
  numbersep=5pt,                   
  numberstyle=\tiny\color{blue}, 
  rulecolor=\color{mygray},        
  showspaces=false,               
  showtabs=false,                 
  stepnumber=5,                  
  stringstyle=\color{mymauve},    
  tabsize=3,                      
  title=\lstname                
}

\usepackage[linkcolor=blue,colorlinks=true]{hyperref}
% \usepackage[colorlinks=true,urlcolor=blue]{hyperref}
\hypersetup{citecolor=blue}

\usepackage{cleveref}
\usepackage{siunitx}
\usepackage{bm}

\usepackage[backend=bibtex,style=alphabetic,maxnames=2,natbib=true]{biblatex} % Use the bibtex backend with the alphabetic citation style (compact APA-like)
% \usepackage[backend=bibtex,style=authoryear,maxnames=2,natbib=true]{biblatex} % Use the bibtex backend with the authoryear citation style (which resembles APA)
\addbibresource{../bib/bibliography.bib} % The filename of the bibliography
\usepackage[autostyle=true]{csquotes} % Required to generate language-dependent quotes in the bibliography 
% \renewcommand*{\bibfont}{\tiny} % Pour reduire la taille des references

\usepackage[useregional=numeric]{datetime2}
\usepackage[normalem]{ulem}

% %-------------------------------------------------------------------------------

\newcommand{\myvec}[3]{\begin{pmatrix} #1  \\ #2 \\ #3 \end{pmatrix}}   %% vecteur 3d
\newcommand{\mymat}[9]{\begin{pmatrix} #1 & #2 & #3 \\ #4 & #5 & #6 \\ #7 & #8 &#9 \end{pmatrix}}  %% Matrice 3*3

\renewcommand{\vector}[4]{\begin{pmatrix} #1  \\ #2 \\ #3 \\ #4 \end{pmatrix}}   %% vecteur 4d
% \newcommand{\mymatrix}[16]{\begin{pmatrix} #1 & #2 & #3 & #4 \\ #4 & #6 & #7 & #8 \\ #9 & #10 & #11 & #12 \\ #13 & #14 & #15 & #16 \end{pmatrix}}  %% Matrice 3*3

\newcommand{\hquad}{\hspace{0.5em}} %% Bew command for half quad
\newcommand*\diff{\mathop{}\!\mathrm{d}}
% \setlength\parindent{0pt}	%% To remove all indentations
\newcommand{\bvec}[1]{\bm{\mathrm{#1}}}  %% Use this to make vectors
\newcommand{\bmat}[1]{\bm{\mathsf{#1}}}   %% Use this to make tensors

%----------------------------------------------------------------------------------------
%	ASSIGNMENT INFORMATION
%----------------------------------------------------------------------------------------

\title{Compte rendu semaine \#5} % Assignment title
% \title{Difficultés rencontrées} % Assignment title

\author{Roussel Desmond Nzoyem} % Student name

\date{\DTMdisplaydate{2021}{3}{3}{-1} - \DTMdisplaydate{2021}{3}{9}{-1}} % Due date

\institute{Sorbonne Université \\ Laboratoire Jacques-Louis Lions} % Institute or school name

\class{Stage M2} % Course or class name

\professor{Pr. Stéphane Labbé} % Professor or teacher in charge of the assignment

%----------------------------------------------------------------------------------------

\begin{document}

\maketitle % Output the assignment title, created automatically using the information in the custom commands above

%----------------------------------------------------------------------------------------
%	ASSIGNMENT CONTENT - INTRO
%----------------------------------------------------------------------------------------


Cette semaine ...


%----------------------------------------------------------------------------------------
%	ASSIGNMENT CONTENT - SECTION 1
%----------------------------------------------------------------------------------------

\section{Tâches effectuées}

\textit{Dans l'ordre chronologique, voici les taches que j'ai effectuées :}
\begin{enumerate}
  \item Lecture du chapitre 2 du livre \parencite{chiu2013stochastic} cité à plusieurs reprises dans le chapitre 4 de la thèse \parencite{balasoiu2020halthesis}. 
  \item Résivions de notions de théorie de la mesure: \textbf{mesure et formules de Campbell}, \textbf{distribution de Palm}, etc.
\end{enumerate}

%----------------------------------------------------------------------------------------
%	ASSIGNMENT CONTENT - SECTION 2
%----------------------------------------------------------------------------------------

\section{Difficultés rencontrées}

\textit{La majorité de ces questions concernent la proposition 3.3.1 de \parencite[p.93]{balasoiu2020halthesis}.}

\begin{enumerate}
  \item Dans le chapitre 4 \parencite[p.113]{balasoiu2020halthesis}, le processus stochastique $\Phi$ est défini comme une variable aléatoire (i.e. une fonction de $\mathcal{A}$ vers $\mathbb{N}$ ?). Lorsqu'on écrit $\text{Card }\Phi$, s'agit-til de  $\text{Card (Im }\Phi)$ ?
  \item Au chapitre 5 \parencite[p.142]{balasoiu2020halthesis}, l'intensité d'un processus de Poisson est défini comme étant un \textbf{scalaire} (comme dans la littérature que j'ai étudié) ; et pourtant dans le chapitre 4 \parencite[p.114]{balasoiu2020halthesis}, l'intensité est une \textbf{mesure}. Comment interpréter cela? 
\end{enumerate}

%----------------------------------------------------------------------------------------
%	ASSIGNMENT CONTENT - SECTION 3
% ----------------------------------------------------------------------------------------

\section{Sujets explorables}

\begin{enumerate}
  \item Dans une prochaine étude, on pourrait étudier la percussion du système masse-
  ressort par un objet solide non ponctuel et qui ne serait pas fixé au système étudié. \citeauthor{balasoiu2020halthesis} pense que le cas général peut se déduire du cas étudié ici. En effet, l’étude de la percussion complète reviendrait à ajouter, dans le système différentiel étudié ici, un nombre fini de perturbations singulières à des instants distincts. 
  \item Il serait également intéressant d’intégrer, comme dans les chapitres 3, 4, et 5, des ressorts de torsion en chaque noeud du système masse-ressort \parencite[p.187]{balasoiu2020halthesis}.
\end{enumerate}

% %-------------------------------------------------------------------------------
% %							THE BIBLIOGRAPHY
% %-------------------------------------------------------------------------------
\clearpage   % Pour retirer les references de la bare de navigation
\printbibliography


\end{document}
