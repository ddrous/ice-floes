%%%%%%%%%%%%%%%%%%%%%%%%%%%%%%%%%%%%%%%%%
% fphw Assignment
% LaTeX Template
% Version 1.0 (27/04/2019)
%
% This template originates from:
% https://www.LaTeXTemplates.com
%
% Authors:
% Class by Felipe Portales-Oliva (f.portales.oliva@gmail.com) with template 
% content and modifications by Vel (vel@LaTeXTemplates.com)
%
% Template (this file) License:
% CC BY-NC-SA 3.0 (http://creativecommons.org/licenses/by-nc-sa/3.0/)
%
%%%%%%%%%%%%%%%%%%%%%%%%%%%%%%%%%%%%%%%%%

%----------------------------------------------------------------------------------------
%	PACKAGES AND OTHER DOCUMENT CONFIGURATIONS
%----------------------------------------------------------------------------------------

\documentclass[
  french,
  % twocolumn,
	11pt, % Default font size, values between 10pt-12pt are allowed
	%letterpaper, % Uncomment for US letter paper size
	%spanish, % Uncomment for Spanish
]{fphw}

% \usepackage[fontsize=10.0]{scrextend} % Use this to force the fontsize

%% Commands for numbering paragraphs
\renewcommand\thesection{\Roman{section}}
\renewcommand\thesubsection{\thesection.\arabic{subsection}}
\renewcommand*\thesubsubsection{%
  \Roman{section}.\arabic{subsection}.\alph{subsubsection}%
}

\usepackage{sectsty}
\sectionfont{\bfseries\Large\raggedright}

% Template-specific packages
\usepackage{babel}
\usepackage[utf8]{inputenc} % Required for inputting international characters
% \usepackage{DejaVuSerifCondensed} 
\usepackage[T1]{fontenc} % Output font encoding for international characters

\usepackage{kpfonts}        %% For math only
\usepackage{fontspec}       %% Because we are using XeTEX
\setromanfont{Minion Pro}   %% For text (Minion Math is commercial)

%-----------------------------------------------------------------------
% \setromanfont{Meta Serif Pro}
% \setsansfont{Fira Sans}
% \setmonofont[Color={0019D4}]{Fira Code} 
%-----------------------------------------------------------------------

\usepackage{fancyvrb}
\usepackage{fvextra}
\newcommand\userinput[1]{\textbf{#1}}
\newcommand\arguments[1]{\textit{#1}}

\usepackage{amsmath}
\usepackage{mathtools}
\usepackage{xfrac} 

\usepackage{graphicx} % Required for including images
\usepackage[textfont=it,font=small]{caption}  %% To manage long captions in images
\usepackage{subcaption}
\captionsetup{justification=centering}

\usepackage{float}
\graphicspath{ {../img/} }

\usepackage{booktabs} % Required for better horizontal rules in tables

\usepackage{listings} % Required for insertion of code

\usepackage{array} % Required for spacing in tabular environment

\usepackage{enumerate} % To modify the enumerate environment

\usepackage{amssymb}
\usepackage{enumitem}	%% % To modify the itemize bullet character

\newcommand{\tabhead}[1]{{\bfseries#1}}

\usepackage{xcolor}
\usepackage{listings}
\colorlet{mygray}{black!30}
\colorlet{mygreen}{green!60!blue}
\colorlet{mymauve}{red!60!blue}
\lstset{
  backgroundcolor=\color{gray!10},  
  basicstyle=\ttfamily,
  columns=fullflexible,
  breakatwhitespace=false,      
  breaklines=true,                
  captionpos=b,                    
  commentstyle=\color{mygreen}, 
  extendedchars=true,              
  frame=single,                   
  keepspaces=true,             
  keywordstyle=\color{blue},      
  language=c++,                 
  numbers=none,                
  numbersep=5pt,                   
  numberstyle=\tiny\color{blue}, 
  rulecolor=\color{mygray},        
  showspaces=false,               
  showtabs=false,                 
  stepnumber=5,                  
  stringstyle=\color{mymauve},    
  tabsize=3,                      
  title=\lstname                
}

\usepackage[linkcolor=blue,colorlinks=true]{hyperref}
% \usepackage[colorlinks=true,urlcolor=blue]{hyperref}
\hypersetup{citecolor=blue}

\usepackage{cleveref}
\usepackage{siunitx}
\usepackage{bm}

\usepackage[backend=bibtex,style=alphabetic,maxnames=2,natbib=true]{biblatex} % Use the bibtex backend with the alphabetic citation style (compact APA-like)
% \usepackage[backend=bibtex,style=authoryear,maxnames=2,natbib=true]{biblatex} % Use the bibtex backend with the authoryear citation style (which resembles APA)
\addbibresource{../bib/bibliography.bib} % The filename of the bibliography
\usepackage[autostyle=true]{csquotes} % Required to generate language-dependent quotes in the bibliography 
% \renewcommand*{\bibfont}{\tiny} % Pour reduire la taille des references

\usepackage[useregional=numeric]{datetime2}
\usepackage[normalem]{ulem}

% %-------------------------------------------------------------------------------

\newcommand{\myvec}[3]{\begin{pmatrix} #1  \\ #2 \\ #3 \end{pmatrix}}   %% vecteur 3d
\newcommand{\mymat}[9]{\begin{pmatrix} #1 & #2 & #3 \\ #4 & #5 & #6 \\ #7 & #8 &#9 \end{pmatrix}}  %% Matrice 3*3

\renewcommand{\vector}[4]{\begin{pmatrix} #1  \\ #2 \\ #3 \\ #4 \end{pmatrix}}   %% vecteur 4d
% \newcommand{\mymatrix}[16]{\begin{pmatrix} #1 & #2 & #3 & #4 \\ #4 & #6 & #7 & #8 \\ #9 & #10 & #11 & #12 \\ #13 & #14 & #15 & #16 \end{pmatrix}}  %% Matrice 3*3

\newcommand{\hquad}{\hspace{0.5em}} %% Bew command for half quad
\newcommand*\diff{\mathop{}\!\mathrm{d}}
% \setlength\parindent{0pt}	%% To remove all indentations
\newcommand{\bvec}[1]{\bm{\mathrm{#1}}}  %% Use this to make vectors
\newcommand{\bmat}[1]{\bm{\mathsf{#1}}}   %% Use this to make tensors

%----------------------------------------------------------------------------------------
%	ASSIGNMENT INFORMATION
%----------------------------------------------------------------------------------------

\title{Compte rendu semaine \#7} % Assignment title
% \title{Difficultés rencontrées} % Assignment title

\author{Roussel Desmond Nzoyem} % Student name

\date{\DTMdisplaydate{2021}{3}{17}{-1} - \DTMdisplaydate{2021}{3}{23}{-1}} % Due date

\institute{Sorbonne Université \\ Laboratoire Jacques-Louis Lions} % Institute or school name

\class{Stage M2} % Course or class name

\professor{Pr. Stéphane Labbé} % Professor or teacher in charge of the assignment

%----------------------------------------------------------------------------------------

\begin{document}

\maketitle % Output the assignment title, created automatically using the information in the custom commands above

%----------------------------------------------------------------------------------------
%	ASSIGNMENT CONTENT - INTRO
%----------------------------------------------------------------------------------------


Le travail durant cette semaine a consisté en la définition d'un modèle décrivant la collision singulière de deux floes de glace. Pour ce faire, il a fallu suivre les indications dans \parencite{balasoiu2020halthesis} (et tester le code de calcul \verb|simuressorts|); ensuite étudier l'état de l'art pour des problèmes de collision avec des réseaux de ressort; et ensuite écrire un modèle se basant sur la thèse \parencite{rabatel2015thesis}.




%----------------------------------------------------------------------------------------
%	ASSIGNMENT CONTENT - SECTION 1
%----------------------------------------------------------------------------------------

\section*{Tâches effectuées}

\begin{enumerate}
  \item Test et debuggage du code \verb|spingslattice| sur \href{https://framagit.org/RaK/SimuRessorts}{Framagit}. J'ai pu observer quelques simulations avec l'interface CLI. Cependant, malgrès l'aide de Dimitri, je n'ai pas pu faire usage de l'interface WEB.
  \item Étude de l'état de l'art pour des problèmes de collision utilisant des réseaux de ressorts. J'ai lu par example \parencite{islam2020numerical,gerivani2019proposing,homodeling,manea2021simplified}, cependant ces modèle sont trop "simples" et entrent difficilement dans la continuité du travail effectué par \citeauthor{balasoiu2020halthesis}.
  \item En relisant la section 1.1.2 du chapitre 1 de la thèse de \citeauthor{rabatel2015thesis} \parencite[p.18]{rabatel2015thesis}, j'ai pensé que pour l'étude de la collision singulière\footnote{Contact en un seul noeud du réseau de ressorts. La percussion incluant plusieurs contact successifs sera étudiée plus tard.}, on pourrait tranformer le système extérieur (SE) de \parencite[p.188]{balasoiu2020halthesis} en problèmes lineaires de complémentarité\footnote{Une pour la condition de non-interpénétration, et une autre pour la loi de frinction de Coulomb.}. Le système intérieur (SI) reste inchangé durant la durée $\delta t^{*} = t^{+} - t^{-}$ de la collision. Ce modèle permet de découpler les floes à la fin de la percussion. Pour simplifier, j'ai supposé (pour l'instant) que les deux floes se percutent au niveau des noeuds de frontière des réseaux.
  
  % (voir figure ).

  Entre les instants $t^{-}$ et $ t^{+}$, le SE proposé est donc similaire au système 1.1.2 de \parencite[p.35]{rabatel2015thesis}, cette fois appliqué uniquement au noeud de contact $q_0$ du floe $\Omega_j$. Le SI reste delui de \parencite[p.188]{balasoiu2020halthesis}.
  Le but du travail à venir sera de pourver l'exitence d'une solution pour ce système. Une fois celà fait, on pourra simuler numériquement la solution (limite quasi-statique à grande raideur) afin de connaitre le déplacemnet sur le bord du floe, et appliquer les résultats du chapitre 2 de \parencite{balasoiu2020halthesis} pour extraire le chemin suivi par la fracture.
\end{enumerate}

%----------------------------------------------------------------------------------------
%	ASSIGNMENT CONTENT - SECTION 2
%----------------------------------------------------------------------------------------

\section*{Difficultés rencontrées}


\begin{enumerate}
  \item J'hésite à intégrer les phases de compression et de décompression \footnote{Les deux phases de la loi de Poisson qui assure la dissipation de l'énergie dans ce contact inélastique.} dans le modèle car ca rendrait le modèle trop lourd. Cette approche déjà pourrait t-elle marcher ?
  \item Les simulations des processus stochatiques du dernier chapitre de \parencite{balasoiu2020halthesis}. En effet, l'interface WEB qui aurait faciliter l'utilisation du programme ne fonctionne pas comme prévu. J'ai donc décider d'utiliser l'interface CLI. Cependant, je me demande: comment est-ce que ces résultats pourraient etre utiles dans les travaux de percusion singulière à venir ?
\end{enumerate}

%----------------------------------------------------------------------------------------
%	ASSIGNMENT CONTENT - SECTION 3
% ----------------------------------------------------------------------------------------

% \section*{Sujets explorables}



% %-------------------------------------------------------------------------------
% %							THE BIBLIOGRAPHY
% %-------------------------------------------------------------------------------
\clearpage   % Pour retirer les references de la bare de navigation
\printbibliography


\end{document}
