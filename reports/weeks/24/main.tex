%%%%%%%%%%%%%%%%%%%%%%%%%%%%%%%%%%%%%%%%%
% fphw Assignment
% LaTeX Template
% Version 1.0 (27/04/2019)
%
% This template originates from:
% https://www.LaTeXTemplates.com
%
% Authors:
% Class by Felipe Portales-Oliva (f.portales.oliva@gmail.com) with template 
% content and modifications by Vel (vel@LaTeXTemplates.com)
%
% Template (this file) License:
% CC BY-NC-SA 3.0 (http://creativecommons.org/licenses/by-nc-sa/3.0/)
%
%%%%%%%%%%%%%%%%%%%%%%%%%%%%%%%%%%%%%%%%%

%----------------------------------------------------------------------------------------
%	PACKAGES AND OTHER DOCUMENT CONFIGURATIONS
%----------------------------------------------------------------------------------------

\documentclass[
  french,
  % twocolumn,
	11pt, % Default font size, values between 10pt-12pt are allowed
	%letterpaper, % Uncomment for US letter paper size
	%spanish, % Uncomment for Spanish
]{fphw}

% \usepackage[fontsize=10.0]{scrextend} % Use this to force the fontsize

%% Commands for numbering paragraphs
\renewcommand\thesection{\Roman{section}}
\renewcommand\thesubsection{\thesection.\arabic{subsection}}
\renewcommand*\thesubsubsection{%
  \Roman{section}.\arabic{subsection}.\alph{subsubsection}%
}

\usepackage{sectsty}
\sectionfont{\sf\bfseries\LARGE\raggedright}

% Template-specific packages
% \usepackage{babel}


% \renewcommand*\familydefault{\sfdefault}
\usepackage[utf8]{inputenc} % Required for inputting international characters
% \usepackage{DejaVuSerifCondensed} 
\usepackage[T1]{fontenc} % Output font encoding for international characters
\usepackage{textcomp}

\usepackage[lf]{venturis}
% \usepackage[libertine]{newtxmath}
\usepackage{libertinust1math}
\usepackage{bm} 

\usepackage{fancyvrb}
\usepackage{fvextra}
\newcommand\userinput[1]{\textbf{#1}}
\newcommand\arguments[1]{\textit{#1}}

\usepackage{amsmath}
\usepackage{mathtools}
\usepackage{xfrac} 
% \usepackage{amssymb}
% \usepackage{enumitem}	%% % To modify the itemize bullet character 

\usepackage{graphicx} % Required for including images
\usepackage[textfont=it,font=small]{caption}  %% To manage long captions in images
\usepackage{subcaption}
\captionsetup{justification=centering}

\usepackage{float}
\graphicspath{ {../img/} }

\usepackage{booktabs} % Required for better horizontal rules in tables

\usepackage{listings} % Required for insertion of code

\usepackage{array} % Required for spacing in tabular environment

\usepackage{enumerate} % To modify the enumerate environment

\newcommand{\tabhead}[1]{{\bfseries#1}}

\usepackage{xcolor}
\usepackage{listings}
\colorlet{mygray}{black!30}
\colorlet{mygreen}{green!60!blue}
\colorlet{mymauve}{red!60!blue}

\usepackage[linkcolor=blue,colorlinks=true]{hyperref}
% \usepackage[colorlinks=true,urlcolor=blue]{hyperref}
\hypersetup{citecolor=blue}

\usepackage{cleveref}
\usepackage{siunitx}

\usepackage[backend=bibtex,style=alphabetic,maxnames=2,natbib=true]{biblatex} % Use the bibtex backend with the alphabetic citation style (compact APA-like)
% \usepackage[backend=bibtex,style=authoryear,maxnames=2,natbib=true]{biblatex} % Use the bibtex backend with the authoryear citation style (which resembles APA)
% \addbibresource{../bib/bibliography.bib} % The filename of the bibliography
\addbibresource{bibliography.bib} % The filename of the bibliography
\usepackage[autostyle=true]{csquotes} % Required to generate language-dependent quotes in the bibliography 
% \renewcommand*{\bibfont}{\tiny} % Pour reduire la taille des references

\usepackage[useregional=numeric]{datetime2}
\usepackage[normalem]{ulem}

% %-------------------------------------------------------------------------------

\newcommand{\myvec}[3]{\begin{pmatrix} #1  \\ #2 \\ #3 \end{pmatrix}}   %% vecteur 3d
\newcommand{\mymat}[9]{\begin{pmatrix} #1 & #2 & #3 \\ #4 & #5 & #6 \\ #7 & #8 &#9 \end{pmatrix}}  %% Matrice 3*3

\renewcommand{\vector}[4]{\begin{pmatrix} #1  \\ #2 \\ #3 \\ #4 \end{pmatrix}}   %% vecteur 4d
% \newcommand{\mymatrix}[16]{\begin{pmatrix} #1 & #2 & #3 & #4 \\ #4 & #6 & #7 & #8 \\ #9 & #10 & #11 & #12 \\ #13 & #14 & #15 & #16 \end{pmatrix}}  %% Matrice 3*3

\newcommand{\hquad}{\hspace{0.5em}} %% Bew command for half quad
\newcommand*\diff{\mathop{}\!\mathrm{d}}
% \setlength\parindent{0pt}	%% To remove all indentations
\newcommand{\bvec}[1]{\bm{\mathrm{#1}}}  %% Use this to make vectors
\newcommand{\bmat}[1]{\bm{\mathsf{#1}}}   %% Use this to make tensors


%----------------------------------------------------------------------------------------
%	ASSIGNMENT INFORMATION
%----------------------------------------------------------------------------------------

\title{\sf\bfseries Compte rendu semaine \#24} % Assignment title
% \title{Difficultés rencontrées} % Assignment title

\author{Roussel Desmond Nzoyem} % Student name 

\date{\DTMdisplaydate{2021}{7}{14}{-1} - \DTMdisplaydate{2021}{7}{20}{-1}} % Due date

\institute{Sorbonne Université \\ Laboratoire Jacques-Louis Lions} % Institute or school name

\class{Stage M2} % Course or class name

\professor{Pr. Stéphane Labbé} % Professor or teacher in charge of the assignment

%----------------------------------------------------------------------------------------

\begin{document}

\maketitle % Output the assignment title, created automatically using the information in the custom commands above

%----------------------------------------------------------------------------------------
%	ASSIGNMENT CONTENT - INTRO
%----------------------------------------------------------------------------------------


Cette semaine, j'ai essentiellement continué l'étude de la fracture en 1D. Bien que je n'ai pas atteint mon objectif d'implémenter la méthode du champ de phase, j'ai néanmoins avancé car j'ai pu reformuler le problème de percussion 1D de façon à gérer plus de deux floes simultanément. 

%----------------------------------------------------------------------------------------
% ASSIGNMENT CONTENT - SECTION 1
%----------------------------------------------------------------------------------------

\section*{Tâches effectuées}


\begin{enumerate}
  \item Suppression des dictionnaires et adoption d'une approche par classes ; ce qui a considérablement accélérer le développement.
  \item Implémentation des différentes fonctions nécessaires pour simuler le déplacement de plus de deux floes simultanément. \emph{Cette étape est indispensable si on veut réussir la fracture}. J'ai attaché à ce rapport une simulation pour illustrer ce point.
\end{enumerate}


 
%----------------------------------------------------------------------------------------
% ASSIGNMENT CONTENT - SECTION 2
%----------------------------------------------------------------------------------------

\section*{Difficultés rencontrées}


% Les difficultés les plus marquantes sont :
\begin{enumerate}
  \item Utilisation des classes à la place des dictionnaires : ce choix est important parce qu'il permet d'utiliser le code qui a été développé pour les simulations précédentes (percussion, déplacement, etc.). 
  \item La conception de ce problème est d'autant plus compliquée que j'essaie de généraliser le code au maximum. Je veux à tout pris éviter d'avoir à reconcevoir le code entier au cas où il y aurait une fonctionnalité à ajouter.
  \item Gestion de la \textbf{condition de concurrence} pour la modification des attributs des différents noeuds. J'ai posé ce problème la semaine dernière, et même si j'ai réussi à le résoudre lorsqu'on a trois floes, je ne sais pas si la méthode va résister lorsque les floes vont se disloquer.
  \item L'autre problème est la résolution des différents bugs que je rencontre à chaque avancée.  
\end{enumerate} 





%----------------------------------------------------------------------------------------
% ASSIGNMENT CONTENT - SECTION 3
% ----------------------------------------------------------------------------------------
\section*{Travail à venir}

\emph{Par ordre de priorité :}

\begin{enumerate}
  \item Codage des différentes fonctions relatives à la méthode du champ de phase ;
  \item Débogage et test du code ajouté à l'aide de simulations ;
  \item Confirmation du travail par une étude énergétique du système ;
  \item Mise à jour du rapport de stage ;
  \item Début du même travail sur le modèle 2D.
\end{enumerate}



% %-------------------------------------------------------------------------------
% %             THE BIBLIOGRAPHY
% %-------------------------------------------------------------------------------
\clearpage   % Pour retirer les references de la bare de navigation
\printbibliography


\end{document}
