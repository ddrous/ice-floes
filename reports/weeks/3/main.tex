%%%%%%%%%%%%%%%%%%%%%%%%%%%%%%%%%%%%%%%%%
% fphw Assignment
% LaTeX Template
% Version 1.0 (27/04/2019)
%
% This template originates from:
% https://www.LaTeXTemplates.com
%
% Authors:
% Class by Felipe Portales-Oliva (f.portales.oliva@gmail.com) with template 
% content and modifications by Vel (vel@LaTeXTemplates.com)
%
% Template (this file) License:
% CC BY-NC-SA 3.0 (http://creativecommons.org/licenses/by-nc-sa/3.0/)
%
%%%%%%%%%%%%%%%%%%%%%%%%%%%%%%%%%%%%%%%%%

%----------------------------------------------------------------------------------------
%	PACKAGES AND OTHER DOCUMENT CONFIGURATIONS
%----------------------------------------------------------------------------------------

\documentclass[
  french,
  % twocolumn,
	11pt, % Default font size, values between 10pt-12pt are allowed
	%letterpaper, % Uncomment for US letter paper size
	%spanish, % Uncomment for Spanish
]{fphw}

% \usepackage[fontsize=10.0]{scrextend} % Use this to force the fontsize

%% Commands for numbering paragraphs
\renewcommand\thesection{\Roman{section}}
\renewcommand\thesubsection{\thesection.\arabic{subsection}}
\renewcommand*\thesubsubsection{%
  \Roman{section}.\arabic{subsection}.\alph{subsubsection}%
}

% Template-specific packages
\usepackage{babel}
\usepackage[utf8]{inputenc} % Required for inputting international characters
% \usepackage{DejaVuSerifCondensed} 
\usepackage[T1]{fontenc} % Output font encoding for international characters

\usepackage{kpfonts}        %% For math only
\usepackage{fontspec}       %% Because we are using XeTEX
\setromanfont{Minion Pro}   %% For text (Minion Math is commercial)

%-----------------------------------------------------------------------
% \setromanfont{Meta Serif Pro}
% \setsansfont{Fira Sans}
% \setmonofont[Color={0019D4}]{Fira Code} 
%-----------------------------------------------------------------------

\usepackage{fancyvrb}
\usepackage{fvextra}
\newcommand\userinput[1]{\textbf{#1}}
\newcommand\arguments[1]{\textit{#1}}

\usepackage{amsmath}
\usepackage{mathtools}
\usepackage{xfrac} 

\usepackage{graphicx} % Required for including images
\usepackage[textfont=it,font=small]{caption}  %% To manage long captions in images
\usepackage{subcaption}
\captionsetup{justification=centering}

\usepackage{float}
\graphicspath{ {../img/} }

\usepackage{booktabs} % Required for better horizontal rules in tables

\usepackage{listings} % Required for insertion of code

\usepackage{array} % Required for spacing in tabular environment

\usepackage{enumerate} % To modify the enumerate environment

\usepackage{amssymb}
\usepackage{enumitem}	%% % To modify the itemize bullet character

\newcommand{\tabhead}[1]{{\bfseries#1}}

\usepackage{xcolor}
\usepackage{listings}
\colorlet{mygray}{black!30}
\colorlet{mygreen}{green!60!blue}
\colorlet{mymauve}{red!60!blue}
\lstset{
  backgroundcolor=\color{gray!10},  
  basicstyle=\ttfamily,
  columns=fullflexible,
  breakatwhitespace=false,      
  breaklines=true,                
  captionpos=b,                    
  commentstyle=\color{mygreen}, 
  extendedchars=true,              
  frame=single,                   
  keepspaces=true,             
  keywordstyle=\color{blue},      
  language=c++,                 
  numbers=none,                
  numbersep=5pt,                   
  numberstyle=\tiny\color{blue}, 
  rulecolor=\color{mygray},        
  showspaces=false,               
  showtabs=false,                 
  stepnumber=5,                  
  stringstyle=\color{mymauve},    
  tabsize=3,                      
  title=\lstname                
}

\usepackage[linkcolor=blue,colorlinks=true]{hyperref}
% \usepackage[colorlinks=true,urlcolor=blue]{hyperref}
\hypersetup{citecolor=blue}

\usepackage{cleveref}
\usepackage{siunitx}
\usepackage{bm}

\usepackage[backend=bibtex,style=alphabetic,maxnames=2,natbib=true]{biblatex} % Use the bibtex backend with the alphabetic citation style (compact APA-like)
% \usepackage[backend=bibtex,style=authoryear,maxnames=2,natbib=true]{biblatex} % Use the bibtex backend with the authoryear citation style (which resembles APA)
\addbibresource{../bib/bibliography.bib} % The filename of the bibliography
\usepackage[autostyle=true]{csquotes} % Required to generate language-dependent quotes in the bibliography 
% \renewcommand*{\bibfont}{\tiny} % Pour reduire la taille des references

\usepackage[useregional=numeric]{datetime2}
\usepackage[normalem]{ulem}

% %-------------------------------------------------------------------------------

\newcommand{\myvec}[3]{\begin{pmatrix} #1  \\ #2 \\ #3 \end{pmatrix}}   %% vecteur 3d
\newcommand{\mymat}[9]{\begin{pmatrix} #1 & #2 & #3 \\ #4 & #5 & #6 \\ #7 & #8 &#9 \end{pmatrix}}  %% Matrice 3*3

\renewcommand{\vector}[4]{\begin{pmatrix} #1  \\ #2 \\ #3 \\ #4 \end{pmatrix}}   %% vecteur 4d
% \newcommand{\mymatrix}[16]{\begin{pmatrix} #1 & #2 & #3 & #4 \\ #4 & #6 & #7 & #8 \\ #9 & #10 & #11 & #12 \\ #13 & #14 & #15 & #16 \end{pmatrix}}  %% Matrice 3*3

\newcommand{\hquad}{\hspace{0.5em}} %% Bew command for half quad
\newcommand*\diff{\mathop{}\!\mathrm{d}}
% \setlength\parindent{0pt}	%% To remove all indentations
\newcommand{\bvec}[1]{\bm{\mathrm{#1}}}  %% Use this to make vectors
\newcommand{\bmat}[1]{\bm{\mathsf{#1}}}   %% Use this to make tensors

%----------------------------------------------------------------------------------------
%	ASSIGNMENT INFORMATION
%----------------------------------------------------------------------------------------

\title{Compte rendu semaine \#3} % Assignment title
% \title{Difficultés rencontrées} % Assignment title

\author{Desmond Roussel Nzoyem} % Student name

\date{\DTMdisplaydate{2021}{2}{17}{-1} - \DTMdisplaydate{2021}{2}{23}{-1}} % Due date

\institute{Sorbonne Université \\ Laboratoire Jacques-Louis Lions} % Institute or school name

\class{Stage M2} % Course or class name

\professor{Pr. Stéphane Labbé} % Professor or teacher in charge of the assignment

%----------------------------------------------------------------------------------------

\begin{document}

\maketitle % Output the assignment title, created automatically using the information in the custom commands above

%----------------------------------------------------------------------------------------
%	ASSIGNMENT CONTENT - INTRO
%----------------------------------------------------------------------------------------


L'objectif de cette semaine fut 


%----------------------------------------------------------------------------------------
%	ASSIGNMENT CONTENT - SECTION 1
%----------------------------------------------------------------------------------------

\section{Tâches effectuées}

\begin{enumerate}
  \item Révision des notions de mécanique de la rupture (et des éléments de mécanique du solide) à travers le livre \parencite{gross2017fracture} :
  \begin{itemize}
    \item Chapitre 1 section 1.1
    \item Chapitre 1 section 1.2
    \item Chapitre 1 section 1.3 sous-section 1 (sauf \textbf{1.3.1.3 Nonlinear elastic material behavior})
    \item Chapitre 2 section 2.1
  \end{itemize}
\end{enumerate}

%----------------------------------------------------------------------------------------
%	ASSIGNMENT CONTENT - SECTION 2
%----------------------------------------------------------------------------------------

\section{Difficultés rencontrées}

\textbf{\textit{Les questions suivantes sont totalement ou partiellement inspirées de la thèse de D. Balasoiu. Celles qui sont barrées sont celles pour lesquelles une réponse a été trouvée.}}


%----------------------------------------------------------------------------------------
%	ASSIGNMENT CONTENT - SECTION 3
% ----------------------------------------------------------------------------------------

\section{Sujets explorables}

D'après \parencite{balasoiu2020thesis}, ces questions restent ouvertes à l'issue de la thèse :
\begin{enumerate}
  \item Deux des limites souhaitées ont été obtenues, mais il resterait à dériver de l’équation différentielle le comportement du système masse-ressort sur son bord lors d’une percussion.
  \item mesurer le coefficient de restitution du nouveau modèle et de calibrer les nouveaux paramètres physiques comme
  la ténacité de la glace, en menant des expériences dans un bassin réfrigéré où flottent des blocs de glace.
  \item Procéder à des tests dans un bassin à glace en s’appuyant sur le projet européen HYDRALAB+.
  \item Modélisation de phénomènes tels que l’écrasement et la création de \textit{ridges};
  \item Actuellement, les modèles granulaires et continus n’interagissent pas
  entre eux. Il serait intéressant de développer un modèle multi-échelles, en couplant ces deux modèles.
  
\end{enumerate}

% %-------------------------------------------------------------------------------
% %							THE BIBLIOGRAPHY
% %-------------------------------------------------------------------------------
\clearpage   % Pour retirer les references de la bare de navigation
\printbibliography


\end{document}
